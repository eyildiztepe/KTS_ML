% This template is borrowed from the Reed College LaTeX thesis template. Most of the work
% for the document class was done by Sam Noble (SN), as well as this
% template. Later comments etc. by Ben Salzberg (BTS). Additional
% restructuring and APA support by Jess Youngberg (JY).
% Your comments and suggestions are more than welcome; please email
% them to cus@reed.edu
%
% See http://web.reed.edu/cis/help/latex.html for help. There are a
% great bunch of help pages there, with notes on
% getting started, bibtex, etc. Go there and read it if you're not
% already familiar with LaTeX.
%
% Any line that starts with a percent symbol is a comment.
% They won't show up in the document, and are useful for notes
% to yourself and explaining commands.
% Commenting also removes a line from the document;
% very handy for troubleshooting problems. -BTS

% As far as I know, this follows the requirements laid out in
% the 2002-2003 Senior Handbook. Ask a librarian to check the
% document before binding. -SN

%%
%% Preamble
%%
% \documentclass{<something>} must begin each LaTeX document
\documentclass[12pt,twoside]{deuthesis}
% Packages are extensions to the basic LaTeX functions. Whatever you
% want to typeset, there is probably a package out there for it.
% Chemistry (chemtex), screenplays, you name it.
% Check out CTAN to see: http://www.ctan.org/
%%
\usepackage{graphicx,latexsym}
\usepackage{amsmath}
\usepackage{amssymb,amsthm}
\usepackage{longtable,booktabs,setspace}
\usepackage{chemarr} %% Useful for one reaction arrow, useless if you're not a chem major
\usepackage[hyphens]{url}
% Added by CII
\usepackage{hyperref}
\usepackage{lmodern}
\usepackage{float}
\floatplacement{figure}{H}
% End of CII addition
\usepackage{rotating}

% Next line commented out by CII
%%% \usepackage{natbib}
% Comment out the natbib line above and uncomment the following two lines to use the new
% biblatex-chicago style, for Chicago A. Also make some changes at the end where the
% bibliography is included.
%\usepackage{biblatex-chicago}
%\bibliography{thesis}


% Added by CII (Thanks, Hadley!)
% Use ref for internal links
\renewcommand{\hyperref}[2][???]{\autoref{#1}}
\def\chapterautorefname{Chapter}
\def\sectionautorefname{Section}
\def\subsectionautorefname{Subsection}
% End of CII addition

% Added by CII
\usepackage{caption}
\captionsetup{width=5in}
% End of CII addition

% \usepackage{times} % other fonts are available like times, bookman, charter, palatino

% Syntax highlighting #22
  \usepackage{color}
  \usepackage{fancyvrb}
  \newcommand{\VerbBar}{|}
  \newcommand{\VERB}{\Verb[commandchars=\\\{\}]}
  \DefineVerbatimEnvironment{Highlighting}{Verbatim}{commandchars=\\\{\}}
  % Add ',fontsize=\small' for more characters per line
  \usepackage{framed}
  \definecolor{shadecolor}{RGB}{248,248,248}
  \newenvironment{Shaded}{\begin{snugshade}}{\end{snugshade}}
  \newcommand{\AlertTok}[1]{\textcolor[rgb]{0.94,0.16,0.16}{#1}}
  \newcommand{\AnnotationTok}[1]{\textcolor[rgb]{0.56,0.35,0.01}{\textbf{\textit{#1}}}}
  \newcommand{\AttributeTok}[1]{\textcolor[rgb]{0.77,0.63,0.00}{#1}}
  \newcommand{\BaseNTok}[1]{\textcolor[rgb]{0.00,0.00,0.81}{#1}}
  \newcommand{\BuiltInTok}[1]{#1}
  \newcommand{\CharTok}[1]{\textcolor[rgb]{0.31,0.60,0.02}{#1}}
  \newcommand{\CommentTok}[1]{\textcolor[rgb]{0.56,0.35,0.01}{\textit{#1}}}
  \newcommand{\CommentVarTok}[1]{\textcolor[rgb]{0.56,0.35,0.01}{\textbf{\textit{#1}}}}
  \newcommand{\ConstantTok}[1]{\textcolor[rgb]{0.00,0.00,0.00}{#1}}
  \newcommand{\ControlFlowTok}[1]{\textcolor[rgb]{0.13,0.29,0.53}{\textbf{#1}}}
  \newcommand{\DataTypeTok}[1]{\textcolor[rgb]{0.13,0.29,0.53}{#1}}
  \newcommand{\DecValTok}[1]{\textcolor[rgb]{0.00,0.00,0.81}{#1}}
  \newcommand{\DocumentationTok}[1]{\textcolor[rgb]{0.56,0.35,0.01}{\textbf{\textit{#1}}}}
  \newcommand{\ErrorTok}[1]{\textcolor[rgb]{0.64,0.00,0.00}{\textbf{#1}}}
  \newcommand{\ExtensionTok}[1]{#1}
  \newcommand{\FloatTok}[1]{\textcolor[rgb]{0.00,0.00,0.81}{#1}}
  \newcommand{\FunctionTok}[1]{\textcolor[rgb]{0.00,0.00,0.00}{#1}}
  \newcommand{\ImportTok}[1]{#1}
  \newcommand{\InformationTok}[1]{\textcolor[rgb]{0.56,0.35,0.01}{\textbf{\textit{#1}}}}
  \newcommand{\KeywordTok}[1]{\textcolor[rgb]{0.13,0.29,0.53}{\textbf{#1}}}
  \newcommand{\NormalTok}[1]{#1}
  \newcommand{\OperatorTok}[1]{\textcolor[rgb]{0.81,0.36,0.00}{\textbf{#1}}}
  \newcommand{\OtherTok}[1]{\textcolor[rgb]{0.56,0.35,0.01}{#1}}
  \newcommand{\PreprocessorTok}[1]{\textcolor[rgb]{0.56,0.35,0.01}{\textit{#1}}}
  \newcommand{\RegionMarkerTok}[1]{#1}
  \newcommand{\SpecialCharTok}[1]{\textcolor[rgb]{0.00,0.00,0.00}{#1}}
  \newcommand{\SpecialStringTok}[1]{\textcolor[rgb]{0.31,0.60,0.02}{#1}}
  \newcommand{\StringTok}[1]{\textcolor[rgb]{0.31,0.60,0.02}{#1}}
  \newcommand{\VariableTok}[1]{\textcolor[rgb]{0.00,0.00,0.00}{#1}}
  \newcommand{\VerbatimStringTok}[1]{\textcolor[rgb]{0.31,0.60,0.02}{#1}}
  \newcommand{\WarningTok}[1]{\textcolor[rgb]{0.56,0.35,0.01}{\textbf{\textit{#1}}}}

% To pass between YAML and LaTeX the dollar signs are added by CII
\title{MAKİNE ÖĞRENMESİ YAKLAŞIMLARININ KARPAL TÜNEL SENDROMU CİDDİYET SINIFLAMASINDA KULLANILMASI}
%\author{Alper ENGİNAtadeniz SAYARCem GÖRENER} %Tek yazar için
\author{Alper ENGİN \\ Atadeniz SAYAR \\ Cem GÖRENER} %Çok yazar için
% The month and year that you submit your FINAL draft TO THE LIBRARY (May or December)
\date{Mayıs 2022}
\division{İSTATİSTİK BÖLÜMÜ}
\advisor{Dr.~Engin YILDIZTEPE}
\institution{FEN FAKÜLTESİ}
\degree{Bitirme Projesi Raporu}
%If you have two advisors for some reason, you can use the following
% Uncommented out by CII
% End of CII addition

%%% Remember to use the correct department!
\department{İstatistik Bölümü}
% if you're writing a thesis in an interdisciplinary major,
% uncomment the line below and change the text as appropriate.
% check the Senior Handbook if unsure.
%\thedivisionof{The Established Interdisciplinary Committee for}
% if you want the approval page to say "Approved for the Committee",
% uncomment the next line
%\approvedforthe{Committee}

% Added by CII
%%% Copied from knitr
%% maxwidth is the original width if it's less than linewidth
%% otherwise use linewidth (to make sure the graphics do not exceed the margin)
\makeatletter
\def\maxwidth{ %
  \ifdim\Gin@nat@width>\linewidth
    \linewidth
  \else
    \Gin@nat@width
  \fi
}
\makeatother

\renewcommand{\contentsname}{Table of Contents}
% End of CII addition

\setlength{\parskip}{0pt}

% Added by CII

\providecommand{\tightlist}{%
  \setlength{\itemsep}{0pt}\setlength{\parskip}{0pt}}

\Acknowledgements{
Tüm çalışma süresince yönlendiriciliği, katkıları ve yardımları ile yanımızda olan danışmanımız Dr.~Engin YILDIZTEPE 'ye ve böyle bir çalışmayı yapmamız için bize fırsat tanıyan Dokuz Eylül Üniversitesi Fen Fakültesi İstatistik Bölümüne teşekkür ederiz.\\
\strut \\
\strut \\
Alper ENGİN\\
Atadeniz SAYAR\\
Cem GÖRENER\\
}

\Dedication{

}

\Preface{
``MAKİNE ÖĞRENMESİ YAKLAŞIMLARININ KARPAL TÜNEL SENDROMU CİDDİYET SINIFLAMASINDA KULLANILMASI'' başlıklı bitirme projesi raporu tarafımdan okunmuş, kapsamı ve niteliği açısından bir Bitirme Projesi raporu olarak kabul edilmiştir.\\
\strut \\
\strut \\
Dr.~Engin YILDIZTEPE
}

\AbstractTR{
Günümüz yaşam koşullarında çalışma hayatı genellikle ofiste ve bilgisayar başında geçmektedir. Bu tarz işlerin yaygınlaşıp popüler hale gelmesi ile birlikte karpal tünel sendromu (KTS)'nun görülme sıklığı ciddi oranda artmıştır. Bu projede KTS tanıtılmış ve KTS ciddiyet skorlamasında makine öğrenmesi yöntemlerinin kullanımı amaçlı iki uygulama yapılmıştır.\\
\hspace*{0.333em}

\par

Güney Kore'deki bir hastanede yapılan çalışmadan elde edilen 1037 el örneği (Park ve diğerleri, 2021) ile üç sınıflı ve iki sınıfa indirgenmiş veri seti için KTS ciddiyet skorlarının tahminlemesi yapılmıştır. XGBoost, Rassal Ormanlar, K-En Yakın Komşuluk ve Yapay Sinir Ağları sınıflandırma algoritmaları ile modeller kurulduktan sonra ciddiyet skorlarının sınıflandırılması için tahminler yapılmaya çalışılmıştır. Uygulama sonucunda \%82 doğru tahmin oranı ile en iyi sonuçlar iki sınıflı sınıflama durumunda ve XGBoost algoritması ile elde edilmiştir.

~

\textbf{Anahtar Kelimeler:} Karpal Tünel Sendromu (KTS), Çok Sınıflı Sınıflandırma, İki Sınıflı Sınıflandırma, XGBoost, Yapay Sinir Ağları
}

\Abstract{
In today's living conditions, working life is usually spent in the office and in front of the computer. With the spread and popularity of this type of work, the incidence of carpal tunnel syndrome (CTS) has increased significantly. In this project, CTS is introduced, and two applications are made regarding the use of machine learning methods in severity scoring.\\
\hspace*{0.333em}

\par

CTS severity classification estimation is performed for three classes dataset with 1037 hand samples obtained from a study in a hospital in South Korea (Park ve diğerleri, 2021) then the dataset is downgraded to two classes and severity classification estimation is performed one more time. After establishing the models with XGBoost, Random Forests, K-Nearest Neighborhood and Artificial Neural Networks classification algorithms, predictions are made for severity classification. As a result of the application, the best results are obtained in the case of two-class classification and with the XGBoost algorithm with an accurate prediction rate of 82\%.

~

\textbf{Keywords:} Karpal Tunnel Syndrome (CTS), Multiclass Classification, Binary Classification, XGBoost, Artificial Neural Networks
}


	\AtBeginDocument{\renewcommand{\chaptername}{Bölüm}}
 \AtBeginDocument{\renewcommand{\contentsname}{İçerik}}
 \AtBeginDocument{\renewcommand{\listfigurename}{Şekil Listesi}}
 \AtBeginDocument{\renewcommand{\listtablename}{Tablo Listesi}}
 \AtBeginDocument{\renewcommand{\figurename}{Şekil}}
 \AtBeginDocument{\renewcommand{\tablename}{Tablo}}
 \AtBeginDocument{\renewcommand{\appendixname}{Ek}}
% End of CII addition
%%
%% End Preamble
%%
%
\begin{document}

% Everything below added by CII
  \maketitle

\frontmatter % this stuff will be roman-numbered
\pagestyle{empty} % this removes page numbers from the frontmatter
\begin{preface}
	``MAKİNE ÖĞRENMESİ YAKLAŞIMLARININ KARPAL TÜNEL SENDROMU CİDDİYET SINIFLAMASINDA KULLANILMASI'' başlıklı bitirme projesi raporu tarafımdan okunmuş, kapsamı ve niteliği açısından bir Bitirme Projesi raporu olarak kabul edilmiştir.\\
 \strut \\
 \strut \\
 Dr.~Engin YILDIZTEPE
\end{preface}
  \begin{acknowledgements}
    Tüm çalışma süresince yönlendiriciliği, katkıları ve yardımları ile yanımızda olan danışmanımız Dr.~Engin YILDIZTEPE 'ye ve böyle bir çalışmayı yapmamız için bize fırsat tanıyan Dokuz Eylül Üniversitesi Fen Fakültesi İstatistik Bölümüne teşekkür ederiz.\\
    \strut \\
    \strut \\
    Alper ENGİN\\
    Atadeniz SAYAR\\
    Cem GÖRENER\\
  \end{acknowledgements}
\begin{abstractTR}
	Günümüz yaşam koşullarında çalışma hayatı genellikle ofiste ve bilgisayar başında geçmektedir. Bu tarz işlerin yaygınlaşıp popüler hale gelmesi ile birlikte karpal tünel sendromu (KTS)'nun görülme sıklığı ciddi oranda artmıştır. Bu projede KTS tanıtılmış ve KTS ciddiyet skorlamasında makine öğrenmesi yöntemlerinin kullanımı amaçlı iki uygulama yapılmıştır.\\
 \hspace*{0.333em}

 \par

 Güney Kore'deki bir hastanede yapılan çalışmadan elde edilen 1037 el örneği (Park ve diğerleri, 2021) ile üç sınıflı ve iki sınıfa indirgenmiş veri seti için KTS ciddiyet skorlarının tahminlemesi yapılmıştır. XGBoost, Rassal Ormanlar, K-En Yakın Komşuluk ve Yapay Sinir Ağları sınıflandırma algoritmaları ile modeller kurulduktan sonra ciddiyet skorlarının sınıflandırılması için tahminler yapılmaya çalışılmıştır. Uygulama sonucunda \%82 doğru tahmin oranı ile en iyi sonuçlar iki sınıflı sınıflama durumunda ve XGBoost algoritması ile elde edilmiştir.

 ~

 \textbf{Anahtar Kelimeler:} Karpal Tünel Sendromu (KTS), Çok Sınıflı Sınıflandırma, İki Sınıflı Sınıflandırma, XGBoost, Yapay Sinir Ağları
\end{abstractTR}
\begin{abstract}
	In today's living conditions, working life is usually spent in the office and in front of the computer. With the spread and popularity of this type of work, the incidence of carpal tunnel syndrome (CTS) has increased significantly. In this project, CTS is introduced, and two applications are made regarding the use of machine learning methods in severity scoring.\\
 \hspace*{0.333em}

 \par

 CTS severity classification estimation is performed for three classes dataset with 1037 hand samples obtained from a study in a hospital in South Korea (Park ve diğerleri, 2021) then the dataset is downgraded to two classes and severity classification estimation is performed one more time. After establishing the models with XGBoost, Random Forests, K-Nearest Neighborhood and Artificial Neural Networks classification algorithms, predictions are made for severity classification. As a result of the application, the best results are obtained in the case of two-class classification and with the XGBoost algorithm with an accurate prediction rate of 82\%.

 ~

 \textbf{Keywords:} Karpal Tunnel Syndrome (CTS), Multiclass Classification, Binary Classification, XGBoost, Artificial Neural Networks
\end{abstract}

  \hypersetup{linkcolor=black}
  \setcounter{tocdepth}{2}
  \tableofcontents

  \listoftables

  \listoffigures


% This was added by EY
\newlength{\cslhangindent}
\setlength{\cslhangindent}{1.5em}
\newenvironment{CSLReferences}%
  {}%
  {\par}
\newenvironment{cslreferences}%
  {}%
  {\par}

\mainmatter % here the regular arabic numbering starts
\pagestyle{fancyplain} % turns page numbering back on


\hypertarget{giriux15f}{%
\chapter*{Giriş}\label{giriux15f}}
\addcontentsline{toc}{chapter}{Giriş}

Makine öğrenmesi yöntemleri günümüzde hastalıkların belirlenmesinde ve sınıflandırmasında yaygın olarak kullanılmaktadır. Hastalığı belirlemede ve sınıflandırmada kullanılan makine öğrenmesi modelleri eğitim verisine ihtiyaç duymaktadır. Günümüz yaşam koşullarında çalışma hayatı genellikle ofiste ve bilgisayar başında geçmektedir. Bu tarz işlerin yaygınlaşıp popüler hale gelmesi ile birlikte karpal tünel sendromu(KTS)'nun görülme sıklığı ciddi oranda artmıştır. KTS, medyan sinirin karpal tüneli içerisinde baskıya uğraması sonucu ortaya çıkan semptomların genel adıdır. KTS'nin görülme sıkılığı kadınlarda \%3-\%3.4 arasında iken erkeklerde \%0.6-\%2.7 arasında olarak belirlenmiş (Çalıcıoğlu, 2020) ve genelde baskın elde semptomlara rastlanmıştır (Bagatur, 2006). Her ne kadar görülme sıklıkları cinsiyetler bazında farklı olsa bile gözlenme sıklığı yaş ile doğru orantılıdır.
~

~

~

KTS için makine öğrenmesi yöntemleri ile sınıflandırma modelleri kurabilmek ancak sınıflandırıcı ve karar vermeye yarayacak verilerin toplanması ile mümkün olabilir. Hekimler hastanın cerrahi müdahaleye ihtiyacı olup olmadığı belirleyebilmek için tanısal testler yapabilmektedir ve bu test sonuçları makine öğrenmesi modelleri için eğitim verisi olarak kullanılabilir. Bu veriler toplandıktan sonra farklı sınıflandırma algoritmaları ile modeller kurularak sınıflandırma yapılmaktadır. Burada başarılı bir model kurulur ise, hastanın yaptırdığı testleri dikkate alınarak hastalığının ciddiyetini sınıflayabilir ve doktora başvurmasının gerekip gerekmediği kararı verilebilir. Bu yöntemin önemi hastanın erken tanı sayesinde tedavi için zemin hazırlama ve gereksiz girişimsel müdahalelerden kaçınmasıdır. Erken tanı sayesinde hastanın maddi, manevi çıkarları korunurken aynı zamanda acılı olabilecek girişimsel test süreçlerinden kaçınmaları sağlanabilir, hasta bazlı düşünce biçiminin yanı sıra erken tanı sayesinde doktorlar üzerindeki iş yükü de azaltılabilir.

~

~

~

Bu projede Güney Kore'deki bir hastanede yapılan çalışmadan elde edilen 1037 el örneği (Park ve diğerleri, 2021) ile üç (mild, moderate, severe) sınıfa sahip veri seti için KTS ciddiyet sınıflandırma tahminlemesi yapılmıştır. XGBoost, Rassal Ormanlar, K-En Yakın Komşuluk ve Yapay Sinir Ağları sınıflandırma algoritmaları ile modeller kurulduktan sonra ciddiyet sınıflandırması için tahminler yapılmaya çalışılmıştır. Bölüm \ref{KTSTanim}'de karpal tünel hastalığı Epidemiyoloji, Etiyoloji, Semptomlardan bahsedilmiş ve tanı değerlendirilmesi yapılmıştır. Bölüm \ref{yontem}'de ise uygulamada kullanılan sınıflandırma algoritmalarından bahsedilmiştir. Bölüm \ref{uygulama} yapılan uygulamayı içermektedir ve Uygulama 2 bölüme ayrılmıştır. Uygulamanın birinci (\ref{multiclass}) bölümünde üç ayrı sınıf için sınıflandırma sonuçlarına yer verilmiştir. Uygulamanın ikinci (\ref{binary}) bölümünde ise veri üç sınıftan iki sınıfa indirgenmiş ve indirgenmiş verilere ait sınıflandırma sonuçları paylaşılmıştır.

\hypertarget{KTSTanim}{%
\chapter{Karpal Tünel Sendromu}\label{KTSTanim}}

Karpal tünel sendromu, medyan sinirin karpal tüneli içerisinde baskıya uğraması sonucu ortaya çıkan semptomların genel adıdır (Werner ve Andary, 2002).\\
Tarihte ilk kez Pajet tarafından, 1854 yılında medyan sinir hasarının bulguları gözlenirken tanımlanmıştır (Pfeffer, Gelberman, Boyes ve Rydevik, 1988).\\
Karpal tünel sendromu, tanımlanması ve terimleştirilmesi ilk olarak 1947 yılında Brain, Wright ve Wilkinson tarafından yapılmıştır (Love, 1955).
\begin{figure}

{\centering \includegraphics[width=4.72in]{figure/karpal_tunnel} 

}

\caption{Karpal Tünel Anatomisi ve Medyan Sinirin Sıkışması}\label{fig:unnamed-chunk-1}
\end{figure}
\hypertarget{KTSEpidemiyoloji}{%
\section{Epidemiyoloji}\label{KTSEpidemiyoloji}}

KTS prevalansı kadınlarda \%3 ila \%3.4 arasında, erkeklerde ise \%0.6 ila \% 2.7 arasında olarak
belirlenmiştir. İnsidans ise kadınlarda 100.000'de 140, erkeklerde 100.000'de 52 olarak saptanmıştır.
Kadınlarda genellikle menopoz dönemimde sıklıkla görülmüş olsa da hem erkek hem de kadınlarda
gözlenme sıklığı yaş ile doğru orantılıdır (Çalıcıoğlu, 2020). KTS'nin \%40 ila \%60 oranında her iki elde de başlayabileceği çeşitli yayınlarda bildirilmiş olup, iki elde de görüldüğü olgularda baskın elin genellikle semptomları daha önce ve daha şiddetli gösterdiği söylenebilir. KTS tek elde görüldüğü durumlarda ise genellikle semptomlar baskın elde görülür (Bagatur, 2006).

\hypertarget{KTSEtiyoloji}{%
\section{Etiyoloji}\label{KTSEtiyoloji}}

KTS'nin en sık nedeni; herhangi bir etiyolojik etkenin saptanamadığı idiopatik KTS'dir. İdiopatik KTS'de ailesel yatkınlık, obezite, VKİ fazla olması, kare şeklinde bilek yapısı gibi
kişisel faktörlerin etken olduğu düşünülmektedir. Günlük yaşamdaki mekanik etkenler de idiopatik KTS
üzerinde etkin rol oynamaktadır. Montaj işinde çalışan işçiler, fabrika çalışanları, klavye ve bilgisayar
kullananlarda olduğu gibi el bilek fleksiyonun aktif olarak yapıldığı belli hareketlerin çok sık
tekrarlanması da KTS ile ilişkili bulunmuştur (Robbins, 1963).

\hypertarget{KTSSemptom}{%
\section{Semptomlar}\label{KTSSemptom}}

Hastalığın şiddetine bağlı olarak semptomlar değişkendir. Erken evrelerde medyan sinirin duyusal
liflerinin tutulumuna bağlı şikayetler görülür. En yaygın semptom el bileğinin merkezinden uzak
dokularda sızlama ve uyuşuklukla beraber yanıcı tarzda ağrıdır. Başparmak tarafından itibaren ilk üç
parmak ve dördüncü parmağın yanal yarısı etkilenir. Daha ileri dönemlerde el ayasında kas
güçsüzlüğü ve körelme meydana gelir. Bu hastalarda elde, özellikle aktivite ile artan beceriksizlik ve objeleri kavramada kuvvetsizlik görülür (Aroori ve Spence, 2008).

\hypertarget{KTSTani_Ciddi}{%
\section{Tanı ve Ciddiyet Değerlendirmesi}\label{KTSTani_Ciddi}}

KTS'de tanı koymak için hastanın hikayesi, klinik semptomlar, fizik muayene bulguları ve bu bulguları destekleyen çeşitli testler kullanılmaktadır(Ghasemi-Rad ve diğerleri, 2014).
Bu testler elektronörofizyolojik, provokatif testler ve tıbbi görüntülemeye dayanan testlerdir.
Elektronörofizyolojik testler karpal Tünel'e bağlanan elektrotlar ile elektrik sinyallerinin incelenmesi ve sonuçların bilgisayar ile yorumlanmasına dayananan testlerdir.

\begin{figure}

{\centering \includegraphics[width=0.49\linewidth,height=0.18\textheight]{figure/noropati_test} 

}

\caption{Elektronörofizyolojik Test (Kumaş, 2005)}\label{fig:unnamed-chunk-2}
\end{figure}
Provokatif testler hastanın bilek ve parmak eklemlerine fiziksel baskı uygulayacak şekilde birtakım
testler uygulanması ve alınan sonuçların değerlendirmesine dayanan deneysel test yöntemleridir.
Tanısal testler genellikle karpal tüneli görüntülemeye dayanan testlerdir.
\begin{itemize}
\item
  Phalen Testi
  \begin{itemize}
  \tightlist
  \item
    60 saniye boyunca parmaklar ayak ucuna bakacak şekilde el dış yüzleri birleştirilir. Meydan sinir bölgesinde karıncalanma oluşur veya artarsa test pozitiftir.
  \end{itemize}
\item
  Ters Phalen testi
  \begin{itemize}
  \tightlist
  \item
    60 saniye boyunca parmaklar yukarı bakacak şekilde el dış yüzleri birleştirilir. Meydan sinir bölgesinde karıncalanma oluşur veya artarsa test pozitiftir.
  \end{itemize}
\item
  Tinel testi
  \begin{itemize}
  \tightlist
  \item
    Uygulayıcı tarafından karpal tünelin üstüne perküsyon yapılır. Medyan sinir bölgesinde karıncalanma ve elektrik şoku hissi oluşursa test pozitiftir (Kurt, 2020).
  \end{itemize}
\item
  Karpal kompresyon testi
  \begin{itemize}
  \tightlist
  \item
    El bileği düz tutulurken medyan sinirin yakınına başparmak ile bastırılır. Medyan sinir bölgesinde karıncalanma oluşur veya artarsa test pozitiftir.
  \end{itemize}
\item
  Gerilmiş median sinir stres (GMSS) testi
  \begin{itemize}
  \tightlist
  \item
    Medyan sinir hareketliliğinin azaldığı durumlarda medyan sinirin gerilerek lokal iskeminin arttırılması mantığına dayanır.
  \end{itemize}
\end{itemize}
\begin{figure}

{\centering \includegraphics[width=0.49\linewidth,height=0.18\textheight]{figure/phalen} \includegraphics[width=0.49\linewidth,height=0.18\textheight]{figure/tinel} 

}

\caption{Phalen ve Tinel Testi}\label{fig:unnamed-chunk-3}
\end{figure}
\begin{figure}

{\centering \includegraphics[width=0.49\linewidth,height=0.18\textheight]{figure/karpal_komp} \includegraphics[width=0.49\linewidth,height=0.18\textheight]{figure/gerilmis} 

}

\caption{Karpal kompresyon ve GMSS Testi }\label{fig:unnamed-chunk-4}
\end{figure}
Görüntülemeye dayalı testlerde el bileği ve parmakların hareketi sırasında karpal tünel içerisindeki değişiklikleri ve medyan sinirin hareketlerini yorumlayarak hastaya tanı koymayı kolaylaştırır fakat hastalığın şiddeti hakkında bilgi vermez.
\begin{itemize}
\item
  Ultrasonografi
\item
  Düz radyografi
\item
  Bilgisayarlı tomografi
\item
  Manyetik rezonans görüntüleme
\end{itemize}
\begin{figure}

{\centering \includegraphics[width=0.49\linewidth,height=0.22\textheight]{figure/ultraradyo} \includegraphics[width=0.49\linewidth,height=0.22\textheight]{figure/radyog} 

}

\caption{Ultrasonografi ve Düz radyografi }\label{fig:unnamed-chunk-5}
\end{figure}
\begin{figure}

{\centering \includegraphics[width=0.49\linewidth,height=0.22\textheight]{figure/bt} 

}

\caption{Bilgisayarlı tomografi }\label{fig:unnamed-chunk-6}
\end{figure}
İdiopatik KTS'de hastalığın tanımlanmasında Boston Karpal Tünel Sendromu
Anketi (BKTSA) kullanılmaktadır (Levine ve diğerleri, 1993). Bu ankete farklı dillere çevrilmiş ve ülkelere göre uyarlanmıştır. Anketin amacı hastanın yanıtlarına göre bir ciddiyet sınıflandırması yapmaktır. Anketin Türkçe versiyonu Sezgin ve ark. (Sezgin ve diğerleri, 2006) tarafından yayımlanmıştır, ancak BKTSA sadece hastaların verdiği yanıtlara dayanarak bir semptom şiddeti belirlemeyi amaçlar.

Teknolojinin hızla gelişmesi ile birlikte hastalara uygulanan testlerin sonuçlarının toplanmasının kolaylaşmasının yanı sıra testlerin sonuçlarına bağlı olarak hastaya tanı koymak ve tanının şiddetini ve derecesini tespit etmek oldukça kolaylaşmıştır.\\
Makine öğrenmesi ve Yapay zeka uygulamalarının yaygınlaşması ile birlikte bu yöntemlerin tıp alanında da kullanımı artmıştır.

Makine öğrenmesi yöntemlerinin KTS tanısında kullanılmasına örnek olarak.
Ardakani ve ark. (Ardakani ve diğerleri, 2020) tarafından hasta olduğu bilinen kişilerden elde edilen bilgisayarlı tomografi görüntüleri, derin öğrenme metotları kullanılarak başka kişilerin hasta olup olmadığını tespit etmek için kullanılmıştır.\\
Bir diğer çalışma ise 2021 yılında Koyama ve ark. (Koyama ve diğerleri, 2021) tarafından geliştirilen bir mobil uygulama sayesinde hastaların ekranın farklı yerlerinde çıkan cisimlere ulaşma sürelerini baz alarak hastalığın evresini tahminlemeyi amaçlamıştır. Bu uygulama hastanın kendi kendine ev ortamında hastalığına ön tanı koyabilmesi açısından yararlı olabilir.\\
Bunların yanı sıra KTS ciddiyet skoru belirlemek için makine öğrenmesi yöntemlerini kullanan çalışmalar da yapılmaktadır.
Güncel bir çalışmada Park ve ark. (Park ve diğerleri, 2021) 1037 hastadan elde edilen verileri farklı makine öğrenmesi yöntemlerinde kullanarak KTS ciddiyet sınıflandırmasını tahmin etmeyi amaçlamışlardır.

\hypertarget{yontem}{%
\chapter{Yöntem}\label{yontem}}

Bu bölümde uygulama kısmında kullanılan sınıflama algoritmalarına değinilmiştir.

\hypertarget{knn}{%
\section{K - En Yakın Komşuluk Algoritması (K-NN)}\label{knn}}

K-NN algoritması, Cover ve Hart tarafından önerilen, örnek veri noktasının bulunduğu sınıfın ve en yakın komşunun, k değerine göre belirlendiği bir sınıflandırma yöntemidir (Cover ve Hart, 1967).\\
K-NN algoritması, en temel örnek tabanlı öğrenme algoritmaları arasındadır. Örnek tabanlı öğrenme algoritmalarında, öğrenme işlemi eğitim setinde tutulan verilere dayalı olarak gerçekleştirilmektedir. Yeni karşılaşılan bir örnek, eğitim setinde yer alan örnekler ile arasındaki benzerliğe göre sınıflandırılmaktadır (Mitchell ve Learning, 1997). K-NN algoritmasında, eğitim setinde yer alan örnekler n boyutlu sayısal nitelikler ile belirtilir. Her örnek n boyutlu uzayda bir noktayı temsil edecek biçimde tüm eğitim örnekleri n boyutlu bir örnek uzayında tutulur. Bilinmeyen bir örnek ile karşılaşıldığında, eğitim setinden ilgili örneğe en yakın k tane örnek belirlenerek yeni örneğin sınıf etiketi, k en yakın komşusunun sınıf etiketlerinin çoğunluk oylamasına göre atanır (Mining, 2006).

\hypertarget{k-nn-parametleri}{%
\subsection{K-NN Parametleri}\label{k-nn-parametleri}}

K-NN algoritmasında performansı etkileyen 3 adet hiper parametre mevcuttur. Bunlar; Uzaklık ölçütü, komşu sayısı(k) ve ağırlıklandırma yöntemidir.

\hypertarget{uzaklux131k-uxf6luxe7uxfctuxfc}{%
\subsubsection{Uzaklık Ölçütü}\label{uzaklux131k-uxf6luxe7uxfctuxfc}}

En bilinen ve yaygın olarak kullanılan 3 uzaklık;
\begin{itemize}
\tightlist
\item
  Minkowski Uzaklığı\\
\item
  Öklid Uzaklığı\\
\item
  Manhattan Uzaklığı
\end{itemize}
\hypertarget{komux15fu-sayux131sux131-k}{%
\subsubsection{Komşu Sayısı (k)}\label{komux15fu-sayux131sux131-k}}

En yakın komşuluk algoritmasında komşu sayısına (k) göre sınıflama yapıldığından algoritma için en önemli parametresi olduğu söylenebilir. k = 5 olarak belirlendiğinde yeni gözlem kendisine en yakın 5 değer baz alınarak sınıflandırılır.

\hypertarget{aux11fux131rlux131klandux131rma}{%
\subsubsection{Ağırlıklandırma}\label{aux11fux131rlux131klandux131rma}}

Komşular için ağırlık değerleri atanması ile sınıflandırılmakta olan örneğe daha yakın olan komşu örneklerin, çoğunluk oylamasına daha fazla katkı koyması amaçlanır. En çok kullanılan ağırlık değeri atama yöntemleri, her bir komşunun ağırlığının, d, komşular arası uzaklık olmak üzere, 1/d ya da \(1/d^2\) şeklinde alınmasıdır (Doad ve Bartere, 2013).

\hypertarget{random_forest}{%
\section{Rassal Ormanlar}\label{random_forest}}

Rassal ormanlar sınıflandırıcısı, her biri farklı girdi vektörlerinden oluşan ve her ağacın yalnızca bir sınıfa oy verdiği karar ağaçlarının kombinasyonudur (Leo Breiman, 1999). Rassal ormanlar sınıflandırması ağaç derinliğini büyütmek için her dalda rastgele değişkenleri içerir.

Karar ağaçlarını modellemek, bir budama metodu ve nitelik seçme ölçütü seçmeyi gerektirir. Karar ağacının tüme varımı için kullanınlan bir sürü nitelik seçme yaklaşımı vardır ve çoğu yaklaşımlar niteliğe direkt olarak kalite ölçümü belirler. Karar ağacının tümevarımındaki en sık kullanılan nitelik seçme ölçümleri Information gain ratio kriteri ve Gini indexidir (L. Breiman, Friedman, Olshen ve Stone, t.y.).

Verilen herhangi bir eğitim verisi için gini index o verinin hangi sınıfa ait olduğu olasılığını hesaplar.

\[\begin{aligned}
\sum \sum_{j \neq i}(f(C_{i}, T) /|T|)(f(C_{j}, T) /|T|)
\end{aligned}\]
Bu denklemde \(f(C_{i}, T) /|T|)\) seçilen gözlemin \(C_{i}\) sınıfa ait olma olasılığıdır.

Her seferinde bir ağaç, değişkenlerin bir kombinasyonunu kullanarak yeni eğitim verisi üzerinde en büyük derinliğe kadar büyür. Son derinliğine ulaşmış bu ağaçlar budanmamıştır. Bu durum, (Quinlan, 2014) veya diğer karar ağacı metotlarına göre rassal ormanlanların en büyük avantajıdır. Bazı durumlarda maliyet ve karmaşıklığı minimum yapabilmek için rassal ormanlar içerisindeki karar ağaçlarına budama yapılır. Budama parametresi olan `ccp\_alpha' değerinden daha küçük olan en büyük maliyet ve karmaşıklık değerini bulunana kadar karar ağacı budanır.

Sonuç olarak, rastgele ağaçlar metodu, kullanıcının herhangi bir değer belirleyebildiği, N kadar büyüyecek karar ağaçları içerir. Yeni veri setini sınıflandırmak için, seçilen maksimum değişken sayısına göre veri setindeki rastgele olarak paylaştırılmış her gözlem, N kadar karar ağacının her biri tarafından sınıflandırılır. Bu durum için ormanlar en fazla oya sahip sınıfı seçer.\\

\begin{figure}

{\centering \includegraphics[width=1\linewidth,height=0.48\textheight]{figure/rf_example} 

}

\caption{Rassal Ormanlar Model Örneği (TIBCO, 2021)}\label{fig:unnamed-chunk-7}
\end{figure}
\hypertarget{xgboost}{%
\section{XGBoost}\label{xgboost}}

Bu bölümde XGboost algoritmasına ve XGBoost algoritmasının daha iyi anlaşabilmesi için XGBoost'un altyapısını oluşturan gradyan arttırma algoritmasına değinilmiştir.

\hypertarget{gradyan-arttux131rux131mux131}{%
\subsection{Gradyan Arttırımı}\label{gradyan-arttux131rux131mux131}}

Gradyan artırma fikri, Leo Breiman'ın , arttırmanın uygun bir maliyet fonksiyonu üzerinde bir optimizasyon algoritması olarak yorumlanabileceği gözleminden kaynaklanmıştır (Leo Breiman, 1997). Açık regresyon gradyan artırma algoritmaları daha sonra Jerome H. Friedman (Friedman, 2001), tarafından Llew Mason, Jonathan Baxter, Peter Bartlett ve Marcus Frean'ın daha genel fonksiyonel gradyan artırma perspektifiyle eş zamanlı olarak geliştirildi (Mason, Baxter, Bartlett ve Frean, 1999).

Gradyan artırma, genellikle karar ağaçları gibi zayıf tahmin modelleri topluluğu şeklinde bir tahmin modeli üreten, regresyon, sınıflandırma ve diğer görevler için bir makine öğrenimi tekniğidir. Grandyan arrtıma modeli, diğer artırma yöntemlerinde de olduğu gibi aşamalı bir şekilde oluşturulur ve keyfi bir türevlenebilir kayıp fonksiyonun optimizasyonuna izin vererek bunları genelleştirir.\\
Gradyan Arttırımı Algoritması ;
\begin{itemize}
\item
  Adım 1 : \(\{(x_i,y_i)\}_{i=1}^n\) şeklinde veri ve türevlenebilir bir kayıp fonksiyonu \(L(y_i,F(x))\) tanımlanır.
\item
  Adım 2 : \(F_0(x) = \min\limits_{\gamma}\,\sum_{i=1}^{n}L(y_i,\gamma)\) \(\gamma\) = Tahmin Değeri olacak şekilde başlangıç değeri (\(F_0(x)\)) minumum olacak şekilde türevlenip 0'a eşitlenir ve \(F_0(x)\)'in minumum değeri elde edilir.
\item
  Adım 3 :
  \begin{itemize}
  \item
    Adım 3.1 : m=1 \(\to\) M'e kadar bir önceki tahmin değerine göre hatalar hesaplanır. (M burada sınıflandırma veya regresyon ağacı sayısıdır)
    \[r_{im} = -\left[\frac{\partial L(y_i,F(x_i)}{\partial F(x_i)}\right]_{F(x)=F_{m-1}(x)}\:\,i=1 \to n\]
  \item
    Adım 3.2 : \(r_{im}\) değerlerine göre regresyon veya sınıflandırma ağacı eğitilir ve \(R_{jm}\) terminal bölgeleri oluşturulur.
  \item
    Adım 3.3 : Her bir yaprak için çıktı değeri hesaplanır.
    \[\gamma_{jm} = \min\limits_{\gamma}\,\sum_{x_i\in R_{ij}}L(y_i,F_{m-1}(x_i)+\gamma)\]
    Yine aynı şekilde kayıp fonksiyonunun türevi alınır ve değerler toplanıp sınıfa eşitlenir. Çıkan sayı yaprağın değeridir.
  \item
    Adım 3.4 : Her gözlem için tahmin oluşturulur.
    \[F_m(x) = F_{m-1}(x) + \nu\sum_{j=1}^{J_m}\gamma_{jm}I(x\in R_{jm})\]
    Formül incelendiğinde yeni ağacın tahmin değerinin önceki ağacın tahmin değeri + (öğrenme düzeyi \(\times\) yeni ağacın değeri) olduğu görülmektedir.
  \end{itemize}
\end{itemize}

\begin{figure}

{\centering \includegraphics[width=1\linewidth,height=0.4\textheight]{figure/gb-dixit} 

}

\caption{Gradyan Arttırma Çalışma Mantığı (Dikker, 2017)}\label{fig:unnamed-chunk-8}
\end{figure}
\hypertarget{xgboost-aux15fux131rux131-gradyan-arttux131rux131mux131}{%
\subsection{XGBoost (Aşırı Gradyan Arttırımı)}\label{xgboost-aux15fux131rux131-gradyan-arttux131rux131mux131}}

XGBoost, temeli gradyan arttırımı ve karar ağacı algoritmalarına dayanan denetimli makine öğrenmesi tekniğidir.
XGBoost algoritmasının orijinal hali Friedman tarafından 2002 yılında geliştirilmiştir (Dikker, 2017). Sonrasında Washington Üniversitesi'nde iki araştırmacı olan Tianqi Chen ve Carlos Guestrin tarafından SIGKDD (Special Interest Group on Knowledge Discovery and Data Mining) 2016 konferansında bildiri olarak sunulmuştur ve makine öğrenme dünyasında çok popüler olmuştur (Chen ve Guestrin, 2016).

XGBoost, yüksek tahmin etme gücüne sahip olup, diğer algoritmalardan daha hızlıdır. Ayrıca XGBoost, genel performansı iyileştiren ve aşırı uyum ya da aşırı öğrenmeyi azaltan bir dizi düzeltme işlemleri içermektedir (Yangın, 2019).

Gradyan arttırımı, güçlü bir sınıflandırıcı oluşturmak için, bir dizi zayıf sınıflandırıcıyı arttırma ile birleştiren topluluk yöntemidir. Güçlü öğrenici, temel bir öğrenici ile başlayarak yinelemeli olarak eğitilmektedir. Hem gradyan arttırımı hem de XGBoost aynı prensibi izlemektedir. Aralarındaki temel farklar uygulama detaylarında yatmaktadır. XGBoost, farklı düzeltme teknikleri kullanarak, ağaçların karmaşıklığını kontrol ederek daha iyi bir performans elde etmeyi başarmaktadır (Salam Patrous, 2018).
\begin{itemize}
\item
  Adım 1 : \(\{(x_i,y_i)\}_{i=1}^n\) şeklinde veri ve türevlenebilir bir kayıp fonksiyonu \(L(y_i,F(x))\) tanımlanır. XGBoost algoritması kayıp fonksiyonu olarak \[L(y_i,P_i) = -[y_{i}log(P_i)+(1-y_i)log(1-P_i)]\] fonksiyonunu kullanır burada \(P_i\) tahmin değerine eşittir.
\item
  Adım 2 : \(F_0(x) = \min\limits_{\gamma}\,\sum_{i=1}^{n}L(y_i,\gamma)\), \(\gamma\) = Tahmin Değeri olacak şekilde başlangıç değeri (\(F_0(x)\)) minumum olacak şekilde türevlenip 0'a eşitlenir ve \(F_0(x)\)'in minumum değeri elde edilir.
\item
  Adım 3 : Her bir yaprak için optimal çıktı değerini elde etmek amacı ile aşağıdaki denklemden yararlanılır.
  \begin{equation}
    \label{eq1}
    L(y,P_i+0_{value}) \approx L(y,P_i) + gO_{value} + \frac{1}{2}hO_{value}^2
  \end{equation}
  \[
  L(y,P_i) = L(y_i,log(Odds)_i) = -y_{i}log(Odds) + log\left(1 + e^{log(Odds)}\right)
  \]
  \[
    g = \left[\frac{d}{dlog(Odds)}L(y_i,log(Odds)_i)\right] = -y_i + \frac{e^{log(Odds)_i}}{1 + e^{log(Odds)_i}} = -(y_i - P_i)
  \]
  \[
    h = \left[\frac{d^2}{dlog(Odds)^2}L(y_i,log(Odds)_i)\right] = \frac{e^{log(Odds)_i}}{(1 + e^{log(Odds)_i})^2} = P_i\times (1-P_i)
  \]
  \begin{itemize}
  \tightlist
  \item
    Adım 3.1 : Denlem \ref{eq1}'den yararlanıldığı takdirde aşağıdaki denklem elde edilmektedir.
    \begin{equation}
      (g_1 + g_2 + \ldots + g_n)O_{value} + \frac{1}{2}(h_1 + h_2 + \ldots + h_n + \lambda)O_{value}^2
      \label{eq2}
      \end{equation}
  \item
    Adım 3.2 : Çıktı değerini elde edebilmek için Denklem \ref{eq2}'de elde edilen denklemin \(O_{value}\)'ya göre türevi alınarak 0'a eşitlenir.
    \[
      \frac{d}{dO_{value}} (g_1 + g_2 + \ldots + g_n)O_{value} + \frac{1}{2}(h_1 + h_2 + \ldots + h_n + \lambda)O_{value}^2 = 0
      \]
    \begin{equation}
      O_{value} = \frac{-(g_1 + g_2 + \ldots + g_n)}{(h_1 + h_2 + \ldots + h_n + \lambda)}
      \label{eq3}
      \end{equation}
  \item
    Adım 3.3 : Elde edilen Denklem \ref{eq3} yardımı ile benzerlik skoru hesaplanabilir. Bu nokta Benzerlik skoru yardımı ile karar noktasına karar verilir. Benzerlik skoru ne kadar düşükse o karar noktasının veri setini daha keskin bir biçimde ayrıştırdığı düşünülür.
    \begin{equation}
      \label{eq4}
      \textrm{Benzerlik Skoru} = \frac{(g_1 + g_2 + \ldots + g_n)^2}{(h_1 + h_2 + \ldots + h_n + \lambda)}
      \end{equation}
  \end{itemize}
\item
  Adım 4 : Sırası ile Denklem \ref{eq4} ve \ref{eq3}, M adet ağaç için hesaplandıktan sonra her ağacın çıktı sonucu eta (öğrenme düzeyi) ile çarpılıp kümülatif olarak toplanır. Son olarak elde edilen ağaçların kümülatif toplamı başta hesaplanan ilk değer ile toplanıp gözlem için tahmin verisi oluşturulur.
\end{itemize}
\begin{figure}

{\centering \includegraphics[width=0.8\linewidth,height=0.3\textheight]{figure/xgb} 

}

\caption{XGBoost Çalışma Mantığı}\label{fig:unnamed-chunk-9}
\end{figure}
\hypertarget{nn}{%
\section{Yapay Sinir Ağları}\label{nn}}

Yapay Sinir ağları insan beyninin en temel özelliği olan öğrenme fonksiyonunu taklit etmeye çalışan sistemlerdir. Öğrenme işlemini örnekler yardımı ile gerçekleştirirler. Bu ağlar birbirine bağlı yapay sinir hücrelerinden oluşur. Yapay sinir ağları biyolojik sinir sisteminden etkilenerek geliştirilmiştir. Biyolojik sinir hücreleri birbirleri ile sinapsisler vasıtası ile iletişim kurarlar. Bir sinir hücresi işlediği bilgileri Axon'ları yolu ile diğer hücrelere gönderirler (Öztemel, 2003).\\

\begin{figure}

{\centering \includegraphics[width=0.8\linewidth,height=0.3\textheight]{figure/basic_neuron} 

}

\caption{Basit Bir Yapay Nöron Örneği (Sahu, 2021)}\label{fig:unnamed-chunk-10}
\end{figure}
Benzer şekilde yapay sinir hücreleri dışarıdan gelen bilgileri bir toplama fonksiyonu ile toplar ve aktivasyon fonksiyonundan geçirerek çıktıyı üretip ağın bağlantılarının üzerinden diğer hücrelere gönderir. Temel bir ağ 3 katmandan meydana gelir;
\begin{itemize}
\item
  Girdi Katmanı
\item
  Ara Katmanlar
\item
  Çıktı Katmanı
\end{itemize}
Her ara katmanın sonunda olabileceği gibi yalnızca çıktı katmanının sonunda da aktivasyon fonksiyonu olabilir. Literatürde en yaygın kullanılan 4 adet aktivasyon fonksiyonu mevcuttur;
\begin{itemize}
\item
  ReLU
\item
  Sigmoid (Lojistik)
\item
  Step
\item
  tanh
\end{itemize}
\begin{figure}

{\centering \includegraphics[width=0.49\linewidth,height=0.18\textheight]{figure/relu} \includegraphics[width=0.49\linewidth,height=0.18\textheight]{figure/sigmoid} 

}

\caption{ReLU ve Sigmoid Fonksiyonu}\label{fig:unnamed-chunk-11}
\end{figure}
\begin{figure}

{\centering \includegraphics[width=0.49\linewidth,height=0.18\textheight]{figure/step} \includegraphics[width=0.49\linewidth,height=0.18\textheight]{figure/tanh} 

}

\caption{Step ve Tanh Fonksiyonu }\label{fig:unnamed-chunk-12}
\end{figure}
\begin{figure}

{\centering \includegraphics[width=1\linewidth,height=0.3\textheight]{figure/cok_katmanli} 

}

\caption{Çok Katmanlı Yapay Sinir Ağı Örneği (Sahu, 2021)}\label{fig:unnamed-chunk-13}
\end{figure}
~
~

Başlangıçta girdi katmanından alınan verilerin ağırlıkları (weights) rastgele olarak atanır ve ağ çalıştırılır. Çıktı katmanından sonra elde edilen sonuç orijinal verinin bağımlı değişkeni ile en yakın değeri üretene kadar sinir ağı geri yayılım yöntemi ile ağırlıkları optimize etmeye çalışır. Maliyet fonksiyonun türevleri alınarak ağırlıkları optimize etmeyi amaçlayan geri yayılım algoritması orijinal verinin bağımlı değişkeni ile tahmin değeri arasında en düşük hata değerini bulduğu zaman ağın yeni ağırlıklarını son bulduğu ağırlıklar olarak tayin eder.
\begin{figure}

{\centering \includegraphics[width=0.85\linewidth,height=0.3\textheight]{figure/geri_yayilim} 

}

\caption{Geri Yaylım Örneği}\label{fig:unnamed-chunk-14}
\end{figure}
Ağın ağırlıkları belirlendikten sonra her bir ağırlığın ne anlama geldiği bilinmemektedir. Bu nedenle yapay sinir ağlarına ``kara kutu'' yakıştırması yapılmaktadır. Ağın performansını etkileyen bağlıca faktörler kullanılan aktivasyon fonksiyonu, öğrenme stratejisi ve öğrenme kuralıdır (Öztemel, 2003).

\hypertarget{uygulama}{%
\chapter{Uygulama}\label{uygulama}}

Bu bölümde yapılan uygulama sunulmuştur. Uygulamada kullanılan veriler 2021 yılında Güney Kore'de yapılan bir araştırmadan alınmıştır (Park ve diğerleri, 2021). Uzmanlar tarafından yapılan tetkikler ile her bir hasta için (her bir el için) ciddiyet sınıflandırılması mild, moderate, severe olarak belirlenmiştir. Çalışmada 1037 elden alınan verilerin 405 (39.05\%) adedi erkek hastalardan, 632 (60.95\%) adedi kadın hastalardan elde edilmiştir. Uzmanlar tarafından 1037 adet elin, 507 (48.9\%) adedi mild, 276 (26.6\%) adedi moderate ve 254 (24.5\%) adedi severe olarak sınıflandırılmıştır.

Araştırmada uzman hekimler tarafından hastalara yöneltilen sorular ile şikayetleri olan ellerine ilişkin aşağıdaki değişkenler için veri toplanmıştır;
\begin{itemize}
\item
  Hands (Eller)
\item
  Age (Yaş)
\item
  Sex (Cinsiyet)
\item
  BMI (Body-mass index, Vücut kitle indeksi)
\item
  Side (Right side involvement, Sağ Elde Bulgu)
\item
  Diabetes (Diyabet Hastalığı Durumu)
\item
  Duration (Duration in months, Dayanma süresi (Aylık))
\item
  NRS (Numeric rating scale of pain, Hissedilen acının numerik karşılığı)
\item
  NP (Noctural Pain, Gece Ağrıları)
\item
  Weakness (Tenar weakness and/or atrophy, Avuç İçi Zayıflık ve/veya Körelme)
\item
  CSA (Cyclosporine dosage in \(mm^2\), Siklosporin dozu (\(mm^2\)))
\item
  PB (fexor retinaculum in mm, Fleksör retinakulum (mm))
\end{itemize}
\begin{longtable}[]{@{}lr@{}}
\caption{\label{tab:overall} Sayısal Değişkenlerin Tanımlayıcı İstatistikleri}\tabularnewline
\toprule
& Overall \\
\midrule
\endfirsthead
\toprule
& Overall \\
\midrule
\endhead
Age,years (mean \(\pm\) SD) & 58 \(\pm\) 10.8 \\
BMI, kg/m\(^2\) (mean \(\pm\) SD) & 24.8 \(\pm\) 3.4 \\
Duration, months (mean \(\pm\) SD) & 8.3 \(\pm\) 9.6 \\
NRS (mean \(\pm\) SD) & 4.4 \(\pm\) 1.8 \\
CSA, mm\(^2\) (mean \(\pm\) SD) & 15.2 \(\pm\) 4.3 \\
PB, mm (mean \(\pm\) SD) & 2.5 \(\pm\) 1.8 \\
\bottomrule
\end{longtable}
\begin{longtable}[]{@{}lrrrr@{}}
\caption{\label{tab:sayisal} Değişkenlerin Bağımlı Değişkene Göre Tanımlayıcı İstatistikleri \footnotesize (P-Value Değerleri Tek Yönlü Varyans Analiz Testi ile Elde Edilmiştir.)}\tabularnewline
\toprule
& Mild & Moderate & Severe & P Value \\
\midrule
\endfirsthead
\toprule
& Mild & Moderate & Severe & P Value \\
\midrule
\endhead
Age,years (mean \(\pm\) SD) & 57.3 \(\pm\) 10.6 & 59.2 \(\pm\) 10.8 & 57.8 \(\pm\) 11.2 & 0.069 \\
BMI, kg/m\(^2\) (mean \(\pm\) SD) & 24.2 \(\pm\) 3.4 & 24.7 \(\pm\) 3 & 25.8 \(\pm\) 3.7 & 0 \\
Duration, months (mean \(\pm\) SD) & 4.3 \(\pm\) 5 & 8.5 \(\pm\) 8.2 & 15.9 \(\pm\) 12.8 & 0 \\
NRS (mean \(\pm\) SD) & 3.3 \(\pm\) 1.3 & 4.9 \(\pm\) 1.5 & 6.1 \(\pm\) 1.5 & 0 \\
CSA, mm\(^2\) (mean \(\pm\) SD) & 13.2 \(\pm\) 3 & 15.4 \(\pm\) 3.2 & 18.9 \(\pm\) 5 & 0 \\
PB, mm (mean \(\pm\) SD) & 2.1 \(\pm\) 0.8 & 2.6 \(\pm\) 2.4 & 3.1 \(\pm\) 2.3 & 0 \\
\bottomrule
\end{longtable}
\hfill\break
\hfill\break
~
\begin{longtable}[]{@{}lrrrr@{}}
\caption{\label{tab:catvar} Katagorik Değişkenlerin Bağımlı Değişkence Frekans Dağılımı \footnotesize (P-Value Değerleri Ki-Kare Bağımsızlık Testi ile Elde Edilmiştir.)}\tabularnewline
\toprule
& Mild & Moderate & Severe & P value \\
\midrule
\endfirsthead
\toprule
& Mild & Moderate & Severe & P value \\
\midrule
\endhead
Eller, n (\%) & 507 (48.9) & 276 (26.6) & 254 (24.5) & - \\
Cinsiyet (Kadın), n (\%) & 308 (60.7) & 153 (55.4) & 171 (67.3) & 0.02 \\
Sağ Elde Bulgu, n (\%) & 243 (47.9) & 149 (54) & 119 (46.9) & 0.181 \\
Diyabet, n (\%) & 47 (9.3) & 45 (16.3) & 54 (21.3) & 0 \\
Gece Ağrıları, n (\%) & 102 (20.1) & 142 (51.4) & 212 (83.5) & 0 \\
Avuç İçi Zayıflık ve/veya Körelme, n (\%) & 1 (0.2) & 24 (8.7) & 169 (66.5) & 0 \\
\bottomrule
\end{longtable}
Kullanılan verilere ait tanımlayıcı istatistikler Tablo \ref{tab:overall}, Tablo \ref{tab:sayisal} ve Tablo \ref{tab:catvar} de verilmiştir. Varsayım kontrollerinin ardından sayısal değişkenlerin ciddiyet sınıflarına göre farklılık gösterip göstermediği tek yönlü varyans analizi ile araştırılmıştır (\(\alpha = 0.05\)). Tablo \ref{tab:sayisal}'de yer alan P-Value değerleri incelendiğinde ciddiyet sınıflamasının yaşa göre anlamlı bir değişim göstermediği diğer tüm değişkenler için anlamlı bir fark olduğu görülmüştür.

Tablo \ref{tab:catvar}'te yer alan P-Value değerleri incelendiğinde ciddiyet sınıflamasının KTS'nin sağ veya sol elde görülmesine göre anlamlı bir değişim göstermediği diğer tüm değişkenler için anlamlı bir fark olduğu görülmüştür.

~
~

Bu çalışmada, KTS ciddiyet sınıflandırması için K-En Yakın Komşuluk, Rassal Ormanlar, Yapay Sinir Ağları ve XGBoost yöntemleri kullanılmıştır.\\
İlk olarak orijinal verilerdeki 3 sınıf (Mild, Moderate, Severe) için sınıflandırma hedeflenmiştir.\\
Bölüm \ref{multiclass}'de bu problem için farklı modeller ile elde edilen sonuçlara yer verilmiştir.\\
Uygulamanın ikinci bölümünde hedef değişken iki sınıfa indirgenmiş ve bu probleme ait sonuçlar bölüm \ref{binary}'de paylaşılmıştır.\\
Uygulamada Python programlama dili ve Scikit-Learn (Buitinck ve diğerleri, 2013), Pandas, Numpy, Matplotlib, XGBoost kütüphanelerinden yararlanılmıştır. Kullanılan kodlar ve veri seti açık kaynaklı olacak şekilde \href{https://github.com/eyildiztepe/KTS_ML}{KTS\_ML} adı altında github deposunda paylaşılmaktadır.
Bu bölümde modellerin performanslarını değerlendirmek üzere kesinlik, duyarlılık, F1-skoru, doğruluk oranı, dengelenmiş doğruluk oranı hesaplanmıştır.

\hypertarget{multiclass}{%
\section{Çok Sınıflı(Multiclass) Sınıflama Problemi}\label{multiclass}}

Bu bölümde KTS ciddiyet sınıflandırması için hedef değişkenin 3 farklı ciddiyet düzeyine sahip olduğu durumda farklı sınıflama algoritmaları ile ciddiyet düzeyinin tahminlenmesi amaçlanmıştır.

\hypertarget{mult_knn}{%
\subsection{K-En Yakın Komşuluk Modeli}\label{mult_knn}}

Bu bölümde veri seti üzerinde K - En yakın komşuluk modeli kullanılmış ve çıktıları değerlendirilmiştir.

\hypertarget{hiper-parametre-seuxe7imi}{%
\subsubsection{Hiper Parametre Seçimi}\label{hiper-parametre-seuxe7imi}}

Daha önce belirlenen parametre uzayını ve Scikit-Learn kütüphanesinde bulunan GridSearchCV algoritması ile en yüksek doğruluk oranı yakalanana kadar çalışması sağlanmıştır.
\begin{figure}

{\centering \includegraphics[width=1.1\linewidth,height=0.55\textheight]{figure/KNN_Grid_Graph} 

}

\caption{Üç Sınıflı K-NN Eğitim Verisi Doğruluk Skorları}\label{fig:unnamed-chunk-21}
\end{figure}
K-En yakın komşuluk modeli için en yüksek doğruluk oranı aşağıdaki parametreler ile bulunmuştur;
\begin{itemize}
\tightlist
\item
  `algorithm':`auto'
\item
  `n\_neighbors':33
\item
  `weights':`distance'
\end{itemize}
\newpage

\hypertarget{en-iyi-parametreli-model}{%
\subsubsection{En İyi Parametreli Model}\label{en-iyi-parametreli-model}}

Bulunan parametrelerle kurulan modelin sınıflandırma metrikleri aşağıdaki gibidir.
\begin{verbatim}
              precision    recall  f1-score   support

        Mild       0.76      0.92      0.83       137
    Moderate       0.51      0.42      0.46        65
      Severe       0.89      0.68      0.77        72

    accuracy                           0.74       274
   macro avg       0.72      0.67      0.69       274
weighted avg       0.73      0.74      0.73       274
\end{verbatim}
\begin{verbatim}
Balanced Accuracy Score : 0.6718827333790838
\end{verbatim}
\begin{figure}

{\centering \includegraphics[width=1.05\linewidth,height=0.6\textheight]{figure/knn_conf} 

}

\caption{Üç Sınıflı K-NN Modeli Karmaşıklık Matrisi}\label{fig:unnamed-chunk-26}
\end{figure}
\begin{figure}

{\centering \includegraphics[width=1.05\linewidth,height=0.6\textheight]{figure/roc_curve_KNeighborsClassifier} 

}

\caption{Üç Sınıflı K-NN Modeli ROC Eğrisi ve AUC Değerleri}\label{fig:unnamed-chunk-27}
\end{figure}
\hypertarget{mult_rf}{%
\subsection{Rassal Ormanlar Modeli}\label{mult_rf}}

Bu bölümde veri seti üzerinde rassal ormanlar modeli kullanılmış ve çıktıları değerlendirilmiştir.

\hypertarget{hiper-parametre-seuxe7imi-1}{%
\subsubsection{Hiper Parametre Seçimi}\label{hiper-parametre-seuxe7imi-1}}

Daha önce belirlenen parametre uzayını ve Scikit-Learn kütüphanesinde bulunan GridSearchCV algoritması ile en yüksek doğruluk oranı yakalanana kadar çalışması sağlanmıştır.
\begin{figure}

{\centering \includegraphics[width=1.1\linewidth,height=0.5\textheight]{figure/RF_Grid_Graph} 

}

\caption{Üç Sınıflı Rassal Ormanlar Modeli Eğitim Verisi Doğruluk Skorları}\label{fig:unnamed-chunk-29}
\end{figure}
Rassal ormanlar modeli için en yüksek doğruluk oranı aşağıdaki parametreler ile bulunmuştur;
\begin{itemize}
\tightlist
\item
  `ccp\_alpha':0.05
\item
  `criterion':`gini'
\item
  `weights':`distance'
\item
  `max\_features':`auto'
\item
  `max\_samples':10
\item
  `n\_estimators':350
\end{itemize}
\newpage

\hypertarget{en-iyi-parametreli-model-1}{%
\subsubsection{En İyi Parametreli Model}\label{en-iyi-parametreli-model-1}}

Bulunan parametrelerle kurulan modelin sınıflandırma metrikleri aşağıdaki gibidir.
\begin{verbatim}
              precision    recall  f1-score   support

        Mild       0.73      0.99      0.84       137
    Moderate       0.58      0.34      0.43        65
      Severe       0.94      0.68      0.79        72

    accuracy                           0.75       274
   macro avg       0.75      0.67      0.69       274
weighted avg       0.75      0.75      0.73       274
\end{verbatim}
\begin{verbatim}
Balanced Accuracy Score : 0.6681395179570361
\end{verbatim}
\begin{figure}

{\centering \includegraphics[width=1.05\linewidth,height=0.6\textheight]{figure/rfc_conf} 

}

\caption{Üç Sınıflı Rassal Ormanlar Modeli Karmaşıklık Matrisi}\label{fig:unnamed-chunk-34}
\end{figure}
\begin{figure}

{\centering \includegraphics[width=1.05\linewidth,height=0.6\textheight]{figure/roc_curve_RandomForestClassifier} 

}

\caption{Üç Sınıflı Rassal Ormanlar Modeli ROC Eğrisi ve AUC Değerleri}\label{fig:unnamed-chunk-35}
\end{figure}
\hypertarget{mult_xgb}{%
\subsection{eXtreme Gradient Boosting (XGBoost)}\label{mult_xgb}}

Bu bölümde veri seti üzerinde XGBoost modeli kullanılmış ve çıktıları değerlendirilmiştir.

\hypertarget{hiper-parametre-seuxe7imi-2}{%
\subsubsection{Hiper Parametre Seçimi}\label{hiper-parametre-seuxe7imi-2}}

Daha önce belirlenen parametre uzayını ve Scikit-Learn kütüphanesinde bulunan GridSearchCV algoritması ile en yüksek doğruluk oranı yakalanana kadar çalışması sağlanmıştır.
\begin{figure}

{\centering \includegraphics[width=1.1\linewidth,height=0.5\textheight]{figure/XGB_Grid_Graph} 

}

\caption{Üç Sınıflı XGBoost Modeli Eğitim Verisi Doğruluk Skorları}\label{fig:unnamed-chunk-37}
\end{figure}
XGBoost modeli için en yüksek doğruluk oranı aşağıdaki parametreler ile bulunmuştur;
\begin{itemize}
\tightlist
\item
  `eta':0.1
\item
  `max\_depth':3
\item
  `min\_child\_weight':10
\item
  `n\_estimators':100
\item
  `objective':`multi:softprob'
\item
  `sumsample':0.5
\end{itemize}
\newpage

Bulunan parametrelerle kurulan modelin sınıflandırma metrikleri aşağıdaki gibidir.
\begin{verbatim}
              precision    recall  f1-score   support

        Mild       0.80      0.89      0.84       137
    Moderate       0.60      0.57      0.58        65
      Severe       0.90      0.74      0.81        72

    accuracy                           0.77       274
   macro avg       0.76      0.73      0.74       274
weighted avg       0.78      0.77      0.77       274
\end{verbatim}
\begin{verbatim}
Balanced Accuracy Score : 0.7319509430823299
\end{verbatim}
\begin{figure}

{\centering \includegraphics[width=1.05\linewidth,height=0.6\textheight]{figure/xgb_conf} 

}

\caption{Üç Sınıflı XGBoost Modeli Karmaşıklık Matrisi}\label{fig:unnamed-chunk-42}
\end{figure}
\begin{figure}

{\centering \includegraphics[width=1.05\linewidth,height=0.6\textheight]{figure/roc_curve_XGBClassifier} 

}

\caption{Üç Sınıflı XGBoost Modeli ROC Eğrisi ve AUC Değerleri}\label{fig:unnamed-chunk-43}
\end{figure}
\hypertarget{mult_nn}{%
\subsection{Yapay Sinir Ağları (Neural Networks)}\label{mult_nn}}

Bu bölümde veri seti üzerinde yapay sinir ağları modeli kullanılmış ve çıktıları değerlendirilmiştir.

\hypertarget{hiper-parametre-seuxe7imi-3}{%
\subsubsection{Hiper Parametre Seçimi}\label{hiper-parametre-seuxe7imi-3}}

Daha önce belirlenen parametre uzayını ve Scikit-Learn kütüphanesinde bulunan GridSearchCV algoritması ile en yüksek doğruluk oranı yakalanana kadar çalışması sağlanmıştır.
\begin{figure}

{\centering \includegraphics[width=1.1\linewidth,height=0.5\textheight]{figure/NN_Grid_Graph} 

}

\caption{Üç Sınıflı Yapay Sinir Ağları Modeli Eğitim Verisi Doğruluk Skorları}\label{fig:unnamed-chunk-45}
\end{figure}
Yapay sinir ağları modeli için en yüksek doğruluk oranı aşağıdaki parametreler ile bulunmuştur;
\begin{itemize}
\tightlist
\item
  `activation':`relu'
\item
  'hidden\_layer\_sizes:19
\item
  `learning\_rate':`adaptive'
\end{itemize}
\newpage

Bulunan parametrelerle kurulan modelin sınıflandırma metrikleri aşağıdaki gibidir.
\begin{verbatim}
              precision    recall  f1-score   support

        Mild       0.83      0.87      0.85       137
    Moderate       0.51      0.58      0.55        65
      Severe       0.89      0.71      0.79        72

    accuracy                           0.76       274
   macro avg       0.75      0.72      0.73       274
weighted avg       0.77      0.76      0.76       274
\end{verbatim}
\begin{verbatim}
Balanced Accuracy Score : 0.7205206188782832
\end{verbatim}
\begin{figure}

{\centering \includegraphics[width=1.05\linewidth,height=0.6\textheight]{figure/nn_conf} 

}

\caption{Üç Sınıflı Yapay Sinir Ağları Modeli Karmaşıklık Matrisi}\label{fig:unnamed-chunk-50}
\end{figure}
\begin{figure}

{\centering \includegraphics[width=1.05\linewidth,height=0.6\textheight]{figure/roc_curve_MLPClassifier} 

}

\caption{Üç Sınıflı Yapay Sinir Ağları Modeli ROC Eğrisi ve AUC Değerleri}\label{fig:unnamed-chunk-51}
\end{figure}
\hypertarget{uxe7ok-sux131nux131flux131-sux131nux131flama-probleminin-modellerinin-deux11ferlendirdirilmesi}{%
\subsection{Çok Sınıflı Sınıflama Probleminin Modellerinin Değerlendirdirilmesi}\label{uxe7ok-sux131nux131flux131-sux131nux131flama-probleminin-modellerinin-deux11ferlendirdirilmesi}}

Bölüm \ref{mult_knn}, \ref{mult_rf}, \ref{mult_xgb} ve \ref{mult_nn}`den elde edilen sonuçlar incelenmiş olup, üç sınıflı problem için \%77 doğru sınıflama oranı ile en iyi model XGBoost olarak bulunmuştur.\\
Bölüm \ref{mult_xgb}'de bulunan performans metrikleri yakından incelendiğinde, duyarlılık metriği 'Moderate' ve `Severe' sınıfları için `Mild' sınıfına kıyasla daha düşük kalmıştır.\\
Duyarlılık metriğindeki düşüklüğün sebep olabileceği yanlış sınıflandırmaların önüne geçebilmek amacı ile bölüm \ref{binary}'de problem iki sınıflı probleme indirgenecek ve modeller tekrar çalıştırılacaktır.

\hypertarget{binary}{%
\section{İki Sınıflı Sınıflama}\label{binary}}

Bu bölümde KTS ciddiyet sınıflandırması için hedef değişkenin 3 farklı ciddiyet düzeyine sahip olduğu veri seti `Moderate' ve `Severe' ciddiyet düzeyleri birleştirilerek problemin 2 farklı ciddiyet düzeyine indirgenip farklı sınıflama algoritmaları ile ciddiyet düzeyinin tahminlenmesi amaçlanmıştır.

\hypertarget{bin_knn}{%
\subsection{K-En Yakın Komşuluk Modeli}\label{bin_knn}}

Bu bölümde veri seti üzerinde K - En yakın komşuluk modeli kullanılmış ve çıktıları değerlendirilmiştir.

\hypertarget{hiper-parametre-seuxe7imi-4}{%
\subsubsection{Hiper Parametre Seçimi}\label{hiper-parametre-seuxe7imi-4}}

Daha önce belirlenen parametre uzayını ve Scikit-Learn kütüphanesinde bulunan GridSearchCV algoritması ile en yüksek doğruluk oranı yakalanana kadar çalışması sağlanmıştır.
\begin{figure}

{\centering \includegraphics[width=1.1\linewidth,height=0.55\textheight]{figure/KNN_bin_Grid_Graph} 

}

\caption{İki Sınıflı K-NN Modeli Eğitim Verisi Doğruluk Skorları}\label{fig:unnamed-chunk-54}
\end{figure}
\newpage

K-En yakın komşuluk modeli için en yüksek doğruluk oranı aşağıdaki parametreler ile bulunmuştur;
\begin{itemize}
\tightlist
\item
  `algorithm':`auto'\\
\item
  `n\_neighbors':23\\
\item
  `weights':`uniform'
\end{itemize}
\hypertarget{en-iyi-parametreli-model-2}{%
\subsubsection{En İyi Parametreli Model}\label{en-iyi-parametreli-model-2}}

Bulunan parametrelerle kurulan modelin sınıflandırma metrikleri aşağıdaki gibidir.
\begin{verbatim}
              precision    recall  f1-score   support

        Mild       0.77      0.88      0.82       153
     Mod+Sev       0.87      0.75      0.80       159

    accuracy                           0.81       312
   macro avg       0.82      0.82      0.81       312
weighted avg       0.82      0.81      0.81       312
\end{verbatim}
\begin{verbatim}
Balanced Accuracy Score : 0.8153903070662227
\end{verbatim}
\begin{figure}

{\centering \includegraphics[width=1.05\linewidth,height=0.6\textheight]{figure/knn_bin_conf} 

}

\caption{İki Sınıflı K-NN Modeli Karmaşıklık Matrisi}\label{fig:unnamed-chunk-59}
\end{figure}
\begin{figure}

{\centering \includegraphics[width=1.05\linewidth,height=0.6\textheight]{figure/KNeighborsClassifier_binary_roc} 

}

\caption{İki Sınıflı K-NN Modeli ROC Eğrisi ve AUC Değerleri}\label{fig:unnamed-chunk-60}
\end{figure}
\hypertarget{bin_rf}{%
\subsection{Rassal Ormanlar Modeli}\label{bin_rf}}

Bu bölümde veri seti üzerinde rassal ormanlar modeli kullanılmış ve çıktıları değerlendirilmiştir.

\hypertarget{hiper-parametre-seuxe7imi-5}{%
\subsubsection{Hiper Parametre Seçimi}\label{hiper-parametre-seuxe7imi-5}}

Daha önce belirlenen parametre uzayını ve Scikit-Learn kütüphanesinde bulunan GridSearchCV algoritması ile en yüksek doğruluk oranı yakalanana kadar çalışması sağlanmıştır.
\begin{figure}

{\centering \includegraphics[width=1.1\linewidth,height=0.55\textheight]{figure/RF_bin_Grid_Graph} 

}

\caption{İki Sınıflı Rassal Ormanlar Modeli Eğitim Verisi Doğruluk Skorları}\label{fig:unnamed-chunk-62}
\end{figure}
Rassal ormanlar modeli için en yüksek doğruluk oranı aşağıdaki parametreler ile bulunmuştur;
\begin{itemize}
\tightlist
\item
  `ccp\_alpha': 0.01\\
\item
  `max\_features': `auto'\\
\item
  `max\_samples': 10\\
\item
  `n\_estimators': 400\\
  \newpage  
\end{itemize}
\hypertarget{en-iyi-parametreli-model-3}{%
\subsubsection{En İyi Parametreli Model}\label{en-iyi-parametreli-model-3}}

Bulunan parametrelerle kurulan modelin sınıflandırma metrikleri aşağıdaki gibidir.
\begin{verbatim}
              precision    recall  f1-score   support

        Mild       0.78      0.85      0.81       153
     Mod+Sev       0.84      0.77      0.80       159

    accuracy                           0.81       312
   macro avg       0.81      0.81      0.81       312
weighted avg       0.81      0.81      0.81       312
\end{verbatim}
\begin{verbatim}
Balanced Accuracy Score : 0.8084844000493279
\end{verbatim}
\begin{figure}

{\centering \includegraphics[width=1.05\linewidth,height=0.6\textheight]{figure/rf_bin_conf} 

}

\caption{İki Sınıflı Rassal Ormanlar Modeli Karmaşıklık Matrisi}\label{fig:unnamed-chunk-67}
\end{figure}
\begin{figure}

{\centering \includegraphics[width=1.05\linewidth,height=0.6\textheight]{figure/RandomForestClassifier_binary_roc} 

}

\caption{İki Sınıflı Rassal Ormanlar Modeli ROC Eğrisi ve AUC Değerleri}\label{fig:unnamed-chunk-68}
\end{figure}
\hypertarget{bin_xgb}{%
\subsection{XGBoost}\label{bin_xgb}}

Bu bölümde veri seti üzerinde XGBoost modeli kullanılmış ve çıktıları değerlendirilmiştir.

\hypertarget{hiper-parametre-seuxe7imi-6}{%
\subsubsection{Hiper Parametre Seçimi}\label{hiper-parametre-seuxe7imi-6}}

Daha önce belirlenen parametre uzayını ve Scikit-Learn kütüphanesinde bulunan GridSearchCV algoritması ile en yüksek doğruluk oranı yakalanana kadar çalışması sağlanmıştır.
\begin{figure}

{\centering \includegraphics[width=1.1\linewidth,height=0.55\textheight]{figure/XGB_bin_Grid_Graph} 

}

\caption{İki Sınıflı XGBoost Modeli Eğitim Verisi Doğruluk Skorları}\label{fig:unnamed-chunk-70}
\end{figure}
XGBoost modeli için en yüksek doğruluk oranı aşağıdaki parametreler ile bulunmuştur;
\begin{itemize}
\tightlist
\item
  `eta': 0.2\\
\item
  `max\_depth': 10\\
\item
  `min\_child\_weight': 10\\
\item
  `n\_estimators': 400\\
  \newpage  
\end{itemize}
\hypertarget{en-iyi-parametreli-model-4}{%
\subsubsection{En İyi Parametreli Model}\label{en-iyi-parametreli-model-4}}

Bulunan parametrelerle kurulan modelin sınıflandırma metrikleri aşağıdaki gibidir.
\begin{verbatim}
              precision    recall  f1-score   support

        Mild       0.82      0.82      0.82       153
     Mod+Sev       0.82      0.83      0.83       159

    accuracy                           0.82       312
   macro avg       0.82      0.82      0.82       312
weighted avg       0.82      0.82      0.82       312
\end{verbatim}
\begin{verbatim}
Balanced Accuracy Score : 0.8235910716487853
\end{verbatim}
\begin{figure}

{\centering \includegraphics[width=1.05\linewidth,height=0.6\textheight]{figure/XGB_bin_conf} 

}

\caption{İki Sınıflı XGBoost Modeli Karmaşıklık Matrisi}\label{fig:unnamed-chunk-75}
\end{figure}
\begin{figure}

{\centering \includegraphics[width=1.05\linewidth,height=0.6\textheight]{figure/XGBClassifier_binary_roc} 

}

\caption{İki Sınıflı XGBoost Modeli ROC Eğrisi ve AUC Değerleri}\label{fig:unnamed-chunk-76}
\end{figure}
\hypertarget{bin_nn}{%
\subsection{Neural Network (Yapay Sinir Ağları) Modeli}\label{bin_nn}}

Bu bölümde veri seti üzerinde yapay sinir ağları modeli kullanılmış ve çıktıları değerlendirilmiştir.

\hypertarget{hiper-parametre-seuxe7imi-7}{%
\subsubsection{Hiper Parametre Seçimi}\label{hiper-parametre-seuxe7imi-7}}

Daha önce belirlenen parametre uzayını ve Scikit-Learn kütüphanesinde bulunan GridSearchCV algoritması ile en yüksek doğruluk oranı yakalanana kadar çalışması sağlanmıştır.
\begin{figure}

{\centering \includegraphics[width=1.1\linewidth,height=0.55\textheight]{figure/NN_bin_Grid_Graph} 

}

\caption{İki Sınıflı Yapay Sinir Ağları Modeli Eğitim Verisi Doğruluk Skorları}\label{fig:unnamed-chunk-78}
\end{figure}
Yapay sinir ağları modeli için en yüksek doğruluk oranı aşağıdaki parametreler ile bulunmuştur;
\begin{itemize}
\tightlist
\item
  `activation': `relu'\\
\item
  `hidden\_layer\_sizes': 19\\
\item
  `learning\_rate': `adaptive'\\
  \newpage  
\end{itemize}
\hypertarget{en-iyi-parametreli-model-5}{%
\subsubsection{En İyi Parametreli Model}\label{en-iyi-parametreli-model-5}}

Bulunan parametrelerle kurulan modelin sınıflandırma metrikleri aşağıdaki gibidir.
\begin{verbatim}
              precision    recall  f1-score   support

        Mild       0.78      0.85      0.82       153
     Mod+Sev       0.84      0.77      0.81       159

    accuracy                           0.81       312
   macro avg       0.81      0.81      0.81       312
weighted avg       0.81      0.81      0.81       312
\end{verbatim}
\begin{verbatim}
Balanced Accuracy Score : 0.8116290541373783
\end{verbatim}
\begin{figure}

{\centering \includegraphics[width=1.05\linewidth,height=0.6\textheight]{figure/nn_bin_conf} 

}

\caption{İki Sınıflı Yapay Sinir Ağları Modeli Karmaşıklık Matrisi}\label{fig:unnamed-chunk-83}
\end{figure}
\begin{figure}

{\centering \includegraphics[width=1.05\linewidth,height=0.6\textheight]{figure/MLPClassifier_binary_roc} 

}

\caption{İki Sınıflı Yapay Sinir Ağları Modeli ROC Eğrisi ve AUC Değerleri}\label{fig:unnamed-chunk-84}
\end{figure}
\hypertarget{iki-sux131nux131flux131-sux131nux131flama-probleminin-modellerinin-deux11ferlendirdirilmesi}{%
\subsection{İki Sınıflı Sınıflama Probleminin Modellerinin Değerlendirdirilmesi}\label{iki-sux131nux131flux131-sux131nux131flama-probleminin-modellerinin-deux11ferlendirdirilmesi}}

Bölüm \ref{bin_knn}, \ref{bin_rf}, \ref{bin_xgb} ve \ref{bin_nn}`den elde edilen sonuçlar incelenmiş olup, iki sınıflı problem için \%82 doğru sınıflama oranları ile XGBoost ve Rassal Ormanlar modelleri en iyi modeller olarak bulunmuştur.\\
Bölüm \ref{bin_rf} ve \ref{bin_xgb}'de bulunan performans metrikleri yakından incelendiğinde, XGBoost modeli kesinlik, duyarlılık ve f1-skor metrikleri bakımından Rassal Ormanlar modelinden daha iyi sonuç vermiştir.\\
Ciddiyet sınıflandırma probleminin 'Mild' ve `Moderate + Severe' olacak şekilde 2 sınıfa indirgendiği durumda XGBoost modeli diğer modellerden daha iyi sonuç vermektedir.

\hypertarget{sonuuxe7}{%
\chapter*{Sonuç}\label{sonuuxe7}}
\addcontentsline{toc}{chapter}{Sonuç}

Bu çalışmada KTS ciddiyet sınıflandırması için makine öğrenmesi yöntemleri kullanılmıştır. Çalışmada Güney Kore'deki bir hastanede yapılan çalışmadan elde edilen ve üç farklı ciddiyet sınıfına sahip (``Mild'', ``Moderate'', ``Severe'') 1037 el örneği (Park ve diğerleri, 2021) üzerinde KTS ciddiyet düzeylerinin tahminlenmesi amacı ile farklı sınıflama modelleri kurulmuş olup bu modellerin kendi aralarında değerlendirmesi yapılmıştır.

~

Uygulamanın ilk bölümünde (\ref{multiclass}) üç sınıflı sınıflandırma problemi için K-En yakın komşuluk, rassal ormanlar, yapay sinir ağları ve XGBoost algoritmaları kullanılmıştır ve \%77 doğru sınıflandırma oranı ile en iyi sonuçlar XGBoost algoritması ile elde edilmiştir. Uygulamanın ikinci bölümünde (\ref{binary}) doğru sınıflandırma oranını arttırabilmek amacı ile ``Mild'' ve ``Moderate + Severe'' olacak şekilde problem iki sınıflı sınıflandırmaya indirgenmiştir. Uygulamanın ikinci bölümünde (\ref{binary}) veri seti üç sınıflıdan iki sınıflıya indirgenmesi ile doğruluk oranlarında ve diğer sınıflama metriklerinde ciddi artışlar gerçekleşmiş ve iki sınıflı problemde XGBoost algoritması KTS ciddiyet skorlarını \%82 oranında doğru sınıflandırmaktadır.

\hfill\break

Çalışmanın sonuçlarına göre KTS ciddiyet skorlarının sınıflanmasında hastalığın maddi, manevi ve zaman kayıpları göz önüne alındığında hastaların bir takım acılı girişimsel testlerden geçmesine engel olabilmek amacı ile makine öğrenmesi yöntemlerinin kullanılabileceği öngörülmektedir. Sonraki çalışmalarda kurulan modelleri temel alan ve hastadan elde edilen verilere göre ciddiyet sınıflaması yapan mobil ya da web uygulamaları düşünülebilir.

\hypertarget{kaynaklar}{%
\chapter*{Kaynaklar}\label{kaynaklar}}
\addcontentsline{toc}{chapter}{Kaynaklar}

\markboth{Kaynaklar}{Kaynaklar}

\hypertarget{refs}{}
\begin{CSLReferences}{1}{0}
\leavevmode\vadjust pre{\hypertarget{ref-ardakani2020diagnosis}{}}%
Ardakani, A. A., Afshar, A., Bhatt, S., Bureau, N. J., Tahmasebi, A., Acharya, U. R. ve Mohammadi, A. (2020). Diagnosis of carpal tunnel syndrome: A comparative study of shear wave elastography, morphometry and artificial intelligence techniques. \emph{Pattern Recognition Letters}, \emph{133}, 77-85.

\leavevmode\vadjust pre{\hypertarget{ref-aroori77carpal}{}}%
Aroori, S. ve Spence, R. (2008). Carpal Tunnel Syndrome. \emph{The Ulster Medical Society}, \emph{77}, 1-17.

\leavevmode\vadjust pre{\hypertarget{ref-avi2022}{}}%
Arora, A. (2022, Mart). 8 unique machine learning interview questions on backpropagation. \emph{Analytics Arora}. \url{https://analyticsarora.com/8-unique-machine-learning-interview-questions-on-backpropagation/} adresinden erişildi.

\leavevmode\vadjust pre{\hypertarget{ref-kts_bagatur}{}}%
Bagatur, A. E. (2006). Karpal Tünel Sendromu. \emph{Türkiye Klinikleri J Surg Med Sci.}, \emph{2}(17), 52-63.

\leavevmode\vadjust pre{\hypertarget{ref-breiman1997arcing}{}}%
Breiman, Leo. (1997). \emph{Arcing the edge}. Technical Report 486, Statistics Department, University of California.

\leavevmode\vadjust pre{\hypertarget{ref-breiman1991}{}}%
Breiman, Leo. (1999). 1 RANDOM FORESTS--RANDOM FEATURES.

\leavevmode\vadjust pre{\hypertarget{ref-breimanclassification}{}}%
Breiman, L., Friedman, J., Olshen, R. ve Stone, C. (t.y.). \emph{Classification and Regression Trees. 1984 Monterey, CA Wadsworth \& Brooks}. \emph{Cole Advanced Books \& Software Google Scholar}.

\leavevmode\vadjust pre{\hypertarget{ref-sklearn_api}{}}%
Buitinck, L., Louppe, G., Blondel, M., Pedregosa, F., Mueller, A., Grisel, O., \ldots{} Varoquaux, G. (2013). {API} design for machine learning software: experiences from the scikit-learn project. \emph{ECML PKDD Workshop: Languages for Data Mining and Machine Learning} içinde (ss. 108-122).

\leavevmode\vadjust pre{\hypertarget{ref-kts_mustafa}{}}%
Çalıcıoğlu, M. N. (2020). \emph{Karpal tünel sendromu olan hastalarda klinik, elektronöromyografik ve ultrasonografik bulguların vücut kitle indeksi ile ilişkisinin değerlendirilmesi}. (Yayımlanmamış mathesis). Hacettepe Üniversitesi.

\leavevmode\vadjust pre{\hypertarget{ref-chen2016xgboost}{}}%
Chen, T. ve Guestrin, C. (2016). Xgboost: A scalable tree boosting system. \emph{Proceedings of the 22nd acm sigkdd international conference on knowledge discovery and data mining} içinde (ss. 785-794).

\leavevmode\vadjust pre{\hypertarget{ref-knn-info}{}}%
Cover, T. ve Hart, P. (1967). Nearest neighbor pattern classification. \emph{IEEE Transactions on Information Theory}, \emph{13}(1), 21-27. doi:\href{https://doi.org/10.1109/TIT.1967.1053964}{10.1109/TIT.1967.1053964}

\leavevmode\vadjust pre{\hypertarget{ref-dikker2017master}{}}%
Dikker, J. (2017). Master thesis Boosted tree learning for balanced item recommendation in online retail.

\leavevmode\vadjust pre{\hypertarget{ref-doad2013review}{}}%
Doad, P. ve Bartere, M. (2013). A Review: Study of Various Clustering Techniques. \emph{International Journal of Engineering Research \& Technology}, \emph{2}(11), 3141-3145.

\leavevmode\vadjust pre{\hypertarget{ref-greedy}{}}%
Friedman, J. H. (2001). Greedy function approximation: A gradient boosting machine. \emph{The Annals of Statistics}, \emph{29}(5).

\leavevmode\vadjust pre{\hypertarget{ref-ghasemi2014handy}{}}%
Ghasemi-Rad, M., Nosair, E., Vegh, A., Mohammadi, A., Akkad, A., Lesha, E., \ldots{} others. (2014). A handy review of carpal tunnel syndrome: From anatomy to diagnosis and treatment. \emph{World journal of radiology}, \emph{6}(6), 284.

\leavevmode\vadjust pre{\hypertarget{ref-hastie_tibshirani_friedman_2009}{}}%
Hastie, T., Tibshirani, R. ve Friedman, J. (2009). \emph{The elements of Statistical Learning, second edition: Data Mining, Inference, and prediction}. Springer.

\leavevmode\vadjust pre{\hypertarget{ref-koyama2021screening}{}}%
Koyama, T., Sato, S., Toriumi, M., Watanabe, T., Nimura, A., Okawa, A., \ldots{} Fujita, K. (2021). A screening method using anomaly detection on a smartphone for patients with carpal tunnel syndrome: Diagnostic case-control study. \emph{JMIR mHealth and uHealth}, \emph{9}(3), e26320.

\leavevmode\vadjust pre{\hypertarget{ref-kumacs2005idiyopatik}{}}%
Kumaş, F. F. (2005). İdiyopatik karpal tÜnel sendromu tedavisinde terapötik ultrason, steroid enjeksiyonu ve splint kullanımının etkinliğinin randimize kontrollü araştırılması.

\leavevmode\vadjust pre{\hypertarget{ref-kurt2020karpal}{}}%
Kurt, A. (2020). \emph{Karpal tünel sendrom hastalarında bilateral ince motor beceri, skapular diskinezi, hareket korkusu ve fonksiyonun sağlıklılarla karşılaştırılması}. (Yayımlanmamış mathesis). Sağlık Bilimleri Enstitüsü.

\leavevmode\vadjust pre{\hypertarget{ref-levine1993self}{}}%
Levine, D. W., Simmons, B. P., Koris, M. J., Daltroy, L. H., Hohl, G. G., Fossel, A. H. ve Katz, J. N. (1993). A self-administered questionnaire for the assessment of severity of symptoms and functional status in carpal tunnel syndrome. \emph{The Journal of bone and joint surgery. American volume}, \emph{75}(11), 1585-1592.

\leavevmode\vadjust pre{\hypertarget{ref-1995love}{}}%
Love, J. (1955). Median neuritis or carpal tunnel syndrome; diagnosis and treatment. \emph{North Carolina medical journal}, \emph{16}(10), 463-469.

\leavevmode\vadjust pre{\hypertarget{ref-mason1999boosting}{}}%
Mason, L., Baxter, J., Bartlett, P. ve Frean, M. (1999). Boosting algorithms as gradient descent. \emph{Advances in neural information processing systems}, \emph{12}.

\leavevmode\vadjust pre{\hypertarget{ref-mining2006data}{}}%
Mining, W. I. D. (2006). Data mining: Concepts and techniques. \emph{Morgan Kaufinann}, \emph{10}, 559-569.

\leavevmode\vadjust pre{\hypertarget{ref-mlmcgraw}{}}%
Mitchell, T. M. ve Learning, M. (1997). McGraw-Hill. \emph{New York}, 154-200.

\leavevmode\vadjust pre{\hypertarget{ref-oztemel2003yapay}{}}%
Öztemel, E. (2003). Yapay sinir ağlari. \emph{PapatyaYayincilik, Istanbul}.

\leavevmode\vadjust pre{\hypertarget{ref-park2021machine}{}}%
Park, D., Kim, B. H., Lee, S.-E., Kim, D. Y., Kim, M., Kwon, H. D., \ldots{} Lee, J. W. (2021). Machine learning-based approach for disease severity classification of carpal tunnel syndrome. \emph{Scientific Reports}, \emph{11}(1), 1-10.

\leavevmode\vadjust pre{\hypertarget{ref-history1988}{}}%
Pfeffer, G., Gelberman, R., Boyes, J. ve Rydevik, B. (1988). The history of carpal tunnnel syndrome. \emph{The Journal of Hand Surgery: British \& European Volume}, \emph{13}(1), 28-34.

\leavevmode\vadjust pre{\hypertarget{ref-quinlan2014c4}{}}%
Quinlan, J. R. (2014). \emph{C4. 5: programs for machine learning}. Elsevier.

\leavevmode\vadjust pre{\hypertarget{ref-robbins1963anatomical}{}}%
Robbins, H. (1963). Anatomical study of the median nerve in the carpal tunnel and etiologies of the carpal-tunnel syndrome. \emph{JBJS}, \emph{45}(5), 953-966.

\leavevmode\vadjust pre{\hypertarget{ref-sahu_2021}{}}%
Sahu, V. (2021, Haziran). Power of a single neuron. \emph{Medium}. Towards Data Science. \url{https://towardsdatascience.com/power-of-a-single-neuron-perceptron-c418ba445095} adresinden erişildi.

\leavevmode\vadjust pre{\hypertarget{ref-salam2018evaluating}{}}%
Salam Patrous, Z. (2018). Evaluating XGBoost for user classification by using behavioral features extracted from smartphone sensors.

\leavevmode\vadjust pre{\hypertarget{ref-sezgi2006assessment}{}}%
Sezgin, M., İncel, N. A., Sevim, S., Çamdeviren, H., As, İ. ve Erdoğan, C. (2006). Assessment of symptom severity and functional status in patients with carpal tunnel syndrome: reliability and validity of the Turkish version of the Boston Questionnaire. \emph{Disability and rehabilitation}, \emph{28}(20), 1281-1286.

\leavevmode\vadjust pre{\hypertarget{ref-tibco}{}}%
TIBCO. (2021). What is a random forest? \emph{TIBCO Software}. \url{https://www.tibco.com/reference-center/what-is-a-random-forest} adresinden erişildi.

\leavevmode\vadjust pre{\hypertarget{ref-werner2002carpal}{}}%
Werner, R. A. ve Andary, M. (2002). Carpal tunnel syndrome: pathophysiology and clinical neurophysiology. \emph{Clinical Neurophysiology}, \emph{113}(9), 1373-1381.

\leavevmode\vadjust pre{\hypertarget{ref-yangin2019xgboost}{}}%
Yangın, G. (2019). \emph{Xgboost ve karar ağaçları tabanlı algoritmaların diyabet veri setleri üzerine uygulaması}. (Yayımlanmamış doktora tezi). Yüksek Lisans Tezi, Mimar Sinan Güzel Sanatlar Üniversitesi Fen Bilimleri.

\end{CSLReferences}
\setlength{\parindent}{-0.20in}
\setlength{\leftskip}{0.20in}
\setlength{\parskip}{8pt}

\appendix

\hypertarget{gerekli-paketlerin-yuxfcklenmesi-verilerin-okunmasux131-ve-veri-uxf6niux15fleme}{%
\chapter{Gerekli Paketlerin Yüklenmesi, Verilerin Okunması ve Veri Önişleme}\label{gerekli-paketlerin-yuxfcklenmesi-verilerin-okunmasux131-ve-veri-uxf6niux15fleme}}

\scriptsize
\begin{Shaded}
\begin{Highlighting}[]
\CommentTok{\# R için gerekli paketlerin kurulması}
\ControlFlowTok{if}\NormalTok{(}\SpecialCharTok{!}\FunctionTok{require}\NormalTok{(reticulate)) }\FunctionTok{install.packages}\NormalTok{(}\StringTok{"reticulate"}\NormalTok{, }\AttributeTok{repos =} \StringTok{"http://cran.rstudio.com"}\NormalTok{)}
\ControlFlowTok{if}\NormalTok{(}\SpecialCharTok{!}\FunctionTok{require}\NormalTok{(tidyverse)) }\FunctionTok{install.packages}\NormalTok{(}\StringTok{"tidyverse"}\NormalTok{, }\AttributeTok{repos =} \StringTok{"http://cran.rstudio.com"}\NormalTok{)}
\ControlFlowTok{if}\NormalTok{(}\SpecialCharTok{!}\FunctionTok{require}\NormalTok{(caret)) }\FunctionTok{install.packages}\NormalTok{(}\StringTok{"caret"}\NormalTok{,}\AttributeTok{repos =} \StringTok{"http://cran.rstudio.com"}\NormalTok{)}
\ControlFlowTok{if}\NormalTok{(}\SpecialCharTok{!}\FunctionTok{require}\NormalTok{(caretEnsemble))  }\FunctionTok{install.packages}\NormalTok{(}\StringTok{"caretEnsemble"}\NormalTok{,}\AttributeTok{repos =} \StringTok{"http://cran.rstudio.com"}\NormalTok{)}
\ControlFlowTok{if}\NormalTok{(}\SpecialCharTok{!}\FunctionTok{require}\NormalTok{(doParallel))  }\FunctionTok{install.packages}\NormalTok{(}\StringTok{"doParallel"}\NormalTok{,}\AttributeTok{repos =} \StringTok{"http://cran.rstudio.com"}\NormalTok{)}
\ControlFlowTok{if}\NormalTok{(}\SpecialCharTok{!}\FunctionTok{require}\NormalTok{(data.table))  }\FunctionTok{install.packages}\NormalTok{(}\StringTok{"data.table"}\NormalTok{,}\AttributeTok{repos =} \StringTok{"http://cran.rstudio.com"}\NormalTok{)}
\ControlFlowTok{if}\NormalTok{(}\SpecialCharTok{!}\FunctionTok{require}\NormalTok{(dplyr))  }\FunctionTok{install.packages}\NormalTok{(}\StringTok{"dplyr"}\NormalTok{,}\AttributeTok{repos =} \StringTok{"http://cran.rstudio.com"}\NormalTok{)}
\ControlFlowTok{if}\NormalTok{(}\SpecialCharTok{!}\FunctionTok{require}\NormalTok{(e1071))  }\FunctionTok{install.packages}\NormalTok{(}\StringTok{"e1071"}\NormalTok{,}\AttributeTok{repos =} \StringTok{"http://cran.rstudio.com"}\NormalTok{)}
\CommentTok{\#if(!require(gbm))  install.packages("gbm",repos = "http://cran.rstudio.com")}
\ControlFlowTok{if}\NormalTok{(}\SpecialCharTok{!}\FunctionTok{require}\NormalTok{(kernlab))  }\FunctionTok{install.packages}\NormalTok{(}\StringTok{"kernlab"}\NormalTok{,}\AttributeTok{repos =} \StringTok{"http://cran.rstudio.com"}\NormalTok{)}
\CommentTok{\#if(!require(randomForest))  install.packages("randomForest",repos = "http://cran.rstudio.com")}
\ControlFlowTok{if}\NormalTok{(}\SpecialCharTok{!}\FunctionTok{require}\NormalTok{(tidyverse))  }\FunctionTok{install.packages}\NormalTok{(}\StringTok{"tidyverse"}\NormalTok{,}\AttributeTok{repos =} \StringTok{"http://cran.rstudio.com"}\NormalTok{)}
\CommentTok{\#if(!require(xgboost))  install.packages("xgboost",repos = "http://cran.rstudio.com")}
\ControlFlowTok{if}\NormalTok{(}\SpecialCharTok{!}\FunctionTok{require}\NormalTok{(smotefamily))  }\FunctionTok{install.packages}\NormalTok{(}\StringTok{"smotefamily"}\NormalTok{,}\AttributeTok{repos =} \StringTok{"http://cran.rstudio.com"}\NormalTok{)}
\end{Highlighting}
\end{Shaded}
\begin{Shaded}
\begin{Highlighting}[]
\CommentTok{\# Python için gerekli paketlerin ve veri setinin yüklenmesi}
\ImportTok{import}\NormalTok{ warnings}
\NormalTok{warnings.filterwarnings(}\StringTok{"ignore"}\NormalTok{, category}\OperatorTok{=}\PreprocessorTok{FutureWarning}\NormalTok{)}
\ImportTok{from}\NormalTok{ warnings }\ImportTok{import}\NormalTok{ simplefilter}
\NormalTok{simplefilter(action}\OperatorTok{=}\StringTok{\textquotesingle{}ignore\textquotesingle{}}\NormalTok{, category}\OperatorTok{=}\PreprocessorTok{FutureWarning}\NormalTok{)}
\ImportTok{import}\NormalTok{ numpy }\ImportTok{as}\NormalTok{ np, pandas }\ImportTok{as}\NormalTok{ pd, matplotlib.pyplot }\ImportTok{as}\NormalTok{ plt}
\ImportTok{import}\NormalTok{ seaborn }\ImportTok{as}\NormalTok{ sns}
\ImportTok{from}\NormalTok{ sklearn.feature\_selection }\ImportTok{import}\NormalTok{ VarianceThreshold}
\ImportTok{from}\NormalTok{ scipy.stats }\ImportTok{import}\NormalTok{ f\_oneway, chi2\_contingency}
\NormalTok{CTS }\OperatorTok{=}\NormalTok{ pd.read\_csv(}\StringTok{"data/CTS.csv"}\NormalTok{,sep}\OperatorTok{=}\StringTok{","}\NormalTok{)}
\NormalTok{dataGroup }\OperatorTok{=}\NormalTok{ CTS[[}\StringTok{"Severity"}\NormalTok{,}\StringTok{"Age"}\NormalTok{,}\StringTok{"BMI"}\NormalTok{,}\StringTok{"CSA"}\NormalTok{,}\StringTok{"PB"}\NormalTok{,}\StringTok{"Duration"}\NormalTok{,}\StringTok{"NRS"}\NormalTok{]]}
\NormalTok{dataOverall }\OperatorTok{=}\NormalTok{ CTS[[}\StringTok{"Age"}\NormalTok{,}\StringTok{"BMI"}\NormalTok{,}\StringTok{"CSA"}\NormalTok{,}\StringTok{"PB"}\NormalTok{,}\StringTok{"Duration"}\NormalTok{,}\StringTok{"NRS"}\NormalTok{]]}
\NormalTok{meanoval, stdoval }\OperatorTok{=} \BuiltInTok{round}\NormalTok{(dataOverall.mean(),}\DecValTok{1}\NormalTok{), }\BuiltInTok{round}\NormalTok{(dataOverall.std(ddof}\OperatorTok{=}\DecValTok{1}\NormalTok{),}\DecValTok{1}\NormalTok{)}
\NormalTok{means }\OperatorTok{=} \BuiltInTok{round}\NormalTok{(dataGroup.groupby(}\StringTok{"Severity"}\NormalTok{).mean(),}\DecValTok{1}\NormalTok{)}
\NormalTok{stds }\OperatorTok{=} \BuiltInTok{round}\NormalTok{(dataGroup.groupby(}\StringTok{"Severity"}\NormalTok{).std(ddof}\OperatorTok{=}\DecValTok{1}\NormalTok{),}\DecValTok{1}\NormalTok{)}
\CommentTok{\#\#}
\NormalTok{mild }\OperatorTok{=}\NormalTok{ CTS[CTS.Severity }\OperatorTok{==} \StringTok{"mild"}\NormalTok{]}
\NormalTok{moderate }\OperatorTok{=}\NormalTok{ CTS[CTS.Severity }\OperatorTok{==} \StringTok{"moderate"}\NormalTok{]}
\NormalTok{severe }\OperatorTok{=}\NormalTok{ CTS[CTS.Severity }\OperatorTok{==} \StringTok{"severe"}\NormalTok{]}
\NormalTok{numVar }\OperatorTok{=}\NormalTok{ [}\StringTok{"Age"}\NormalTok{,}\StringTok{"BMI"}\NormalTok{,}\StringTok{"CSA"}\NormalTok{,}\StringTok{"PB"}\NormalTok{,}\StringTok{"Duration"}\NormalTok{,}\StringTok{"NRS"}\NormalTok{]}
\NormalTok{catVar }\OperatorTok{=}\NormalTok{ [}\StringTok{"Sex"}\NormalTok{,}\StringTok{"Side"}\NormalTok{,}\StringTok{"Diabetes"}\NormalTok{,}\StringTok{"NP"}\NormalTok{,}\StringTok{"Weakness"}\NormalTok{]}
\NormalTok{p\_values }\OperatorTok{=}\NormalTok{ []}
\NormalTok{p\_vals2 }\OperatorTok{=}\NormalTok{ []}
\ControlFlowTok{for}\NormalTok{ i }\KeywordTok{in}\NormalTok{ numVar:}
\NormalTok{    \_,p\_val }\OperatorTok{=}\NormalTok{ f\_oneway(mild[i],moderate[i],severe[i])}
\NormalTok{    p\_values.append(}\BuiltInTok{round}\NormalTok{(p\_val,}\DecValTok{3}\NormalTok{))}
\ControlFlowTok{for}\NormalTok{ i }\KeywordTok{in}\NormalTok{ catVar:}
\NormalTok{    var\_0 }\OperatorTok{=}\NormalTok{ np.array([}\BuiltInTok{sum}\NormalTok{(mild[i] }\OperatorTok{==} \DecValTok{0}\NormalTok{),}\BuiltInTok{sum}\NormalTok{(moderate[i] }\OperatorTok{==} \DecValTok{0}\NormalTok{),}\BuiltInTok{sum}\NormalTok{(severe[i] }\OperatorTok{==} \DecValTok{0}\NormalTok{)])}
\NormalTok{    var\_1 }\OperatorTok{=}\NormalTok{ np.array([}\BuiltInTok{sum}\NormalTok{(mild[i] }\OperatorTok{==} \DecValTok{1}\NormalTok{),}\BuiltInTok{sum}\NormalTok{(moderate[i] }\OperatorTok{==} \DecValTok{1}\NormalTok{),}\BuiltInTok{sum}\NormalTok{(severe[i] }\OperatorTok{==} \DecValTok{1}\NormalTok{)])}
\NormalTok{    p\_vals2.append(}\BuiltInTok{round}\NormalTok{(chi2\_contingency(np.array([var\_1,var\_0]),correction}\OperatorTok{=}\VariableTok{False}\NormalTok{)[}\DecValTok{1}\NormalTok{],}\DecValTok{3}\NormalTok{))}
\NormalTok{CTS\_kor }\OperatorTok{=}\NormalTok{ CTS.drop([}\StringTok{"Severity"}\NormalTok{,}\StringTok{"Mild"}\NormalTok{,}\StringTok{"Mod"}\NormalTok{,}\StringTok{"Sev"}\NormalTok{],axis}\OperatorTok{=}\DecValTok{1}\NormalTok{)}
\NormalTok{zeroVar }\OperatorTok{=}\NormalTok{ CTS\_kor.shape[}\DecValTok{1}\NormalTok{]}\OperatorTok{{-}}\NormalTok{((VarianceThreshold(threshold}\OperatorTok{=}\DecValTok{0}\NormalTok{).fit(CTS\_kor)).get\_support()).}\BuiltInTok{sum}\NormalTok{()    }
\CommentTok{\#\#\#\#\#\#}
\NormalTok{catDF }\OperatorTok{=}\NormalTok{ CTS.groupby(}\StringTok{"Severity"}\NormalTok{).}\BuiltInTok{sum}\NormalTok{()[[}\StringTok{"Sex"}\NormalTok{,}\StringTok{"Side"}\NormalTok{,}\StringTok{"Diabetes"}\NormalTok{,}\StringTok{"NP"}\NormalTok{,}\StringTok{"Weakness"}\NormalTok{]]}
\NormalTok{sex, rside, diab, np, weak }\OperatorTok{=}\NormalTok{ catDF[}\StringTok{"Sex"}\NormalTok{],catDF[}\StringTok{"Side"}\NormalTok{],catDF[}\StringTok{"Diabetes"}\NormalTok{],catDF[}\StringTok{"NP"}\NormalTok{],catDF[}\StringTok{"Weakness"}\NormalTok{]}
\NormalTok{hands }\OperatorTok{=}\NormalTok{ CTS.groupby(}\StringTok{"Severity"}\NormalTok{).count()[}\StringTok{"NP"}\NormalTok{]}
\NormalTok{handsx }\OperatorTok{=}\NormalTok{ hands}\OperatorTok{*}\DecValTok{100}
\end{Highlighting}
\end{Shaded}
\begin{Shaded}
\begin{Highlighting}[]
\CommentTok{\# R için veri setinin tanıtılması ve Rastgele olarak ayrılması}
\NormalTok{  means }\OtherTok{\textless{}{-}} \FunctionTok{as.data.frame}\NormalTok{(py}\SpecialCharTok{$}\NormalTok{means)}
\NormalTok{  stds }\OtherTok{\textless{}{-}} \FunctionTok{as.data.frame}\NormalTok{(py}\SpecialCharTok{$}\NormalTok{stds)}
\NormalTok{  meanOval }\OtherTok{\textless{}{-}} \FunctionTok{as.data.frame}\NormalTok{(}\FunctionTok{t}\NormalTok{(py}\SpecialCharTok{$}\NormalTok{meanoval))}
\NormalTok{  stdOval }\OtherTok{\textless{}{-}} \FunctionTok{as.data.frame}\NormalTok{(}\FunctionTok{t}\NormalTok{(py}\SpecialCharTok{$}\NormalTok{stdoval))}
\NormalTok{  CTS }\OtherTok{\textless{}{-}} \FunctionTok{as.data.frame}\NormalTok{(}\FunctionTok{read\_csv}\NormalTok{(}\StringTok{"data/CTS.csv"}\NormalTok{))}
\NormalTok{  seed}\OtherTok{\textless{}{-}}\DecValTok{0923}
  \FunctionTok{set.seed}\NormalTok{(seed)}
\NormalTok{  ind}\OtherTok{\textless{}{-}}\FunctionTok{sample}\NormalTok{(}\DecValTok{2}\NormalTok{,}\FunctionTok{nrow}\NormalTok{(CTS),}\AttributeTok{replace =}\NormalTok{ T,}\AttributeTok{prob =} \FunctionTok{c}\NormalTok{(}\FloatTok{0.7}\NormalTok{,}\FloatTok{0.3}\NormalTok{))}
\NormalTok{  traindata\_top }\OtherTok{\textless{}{-}}\NormalTok{ CTS[ind}\SpecialCharTok{==}\DecValTok{1}\NormalTok{,]}
\NormalTok{  testdata\_top }\OtherTok{\textless{}{-}}\NormalTok{ CTS[ind}\SpecialCharTok{==}\DecValTok{2}\NormalTok{,]}
\CommentTok{\# BURADAN SORNASI R VERİ ÖNİŞLEME}
\NormalTok{  CTS}\SpecialCharTok{$}\NormalTok{Severity}\OtherTok{\textless{}{-}}\FunctionTok{as.factor}\NormalTok{(CTS}\SpecialCharTok{$}\NormalTok{Severity)}
\NormalTok{  CTS}\SpecialCharTok{$}\NormalTok{Mild}\OtherTok{\textless{}{-}}\FunctionTok{as.factor}\NormalTok{(CTS}\SpecialCharTok{$}\NormalTok{Mild)}
\NormalTok{  CTS}\SpecialCharTok{$}\NormalTok{Mod}\OtherTok{\textless{}{-}}\FunctionTok{as.factor}\NormalTok{(CTS}\SpecialCharTok{$}\NormalTok{Mod)}
\NormalTok{  CTS}\SpecialCharTok{$}\NormalTok{Sev}\OtherTok{\textless{}{-}}\FunctionTok{as.factor}\NormalTok{(CTS}\SpecialCharTok{$}\NormalTok{Sev)}
\NormalTok{  CTS}\SpecialCharTok{$}\NormalTok{Sex }\OtherTok{\textless{}{-}}\FunctionTok{as.factor}\NormalTok{(CTS}\SpecialCharTok{$}\NormalTok{Sex)}
\NormalTok{  CTS}\SpecialCharTok{$}\NormalTok{Side }\OtherTok{\textless{}{-}}\FunctionTok{as.factor}\NormalTok{(CTS}\SpecialCharTok{$}\NormalTok{Side)}
\NormalTok{  CTS}\SpecialCharTok{$}\NormalTok{Diabetes }\OtherTok{\textless{}{-}}\FunctionTok{as.factor}\NormalTok{(CTS}\SpecialCharTok{$}\NormalTok{Diabetes)}
\NormalTok{  CTS}\SpecialCharTok{$}\NormalTok{NP }\OtherTok{\textless{}{-}} \FunctionTok{as.factor}\NormalTok{(CTS}\SpecialCharTok{$}\NormalTok{NP)}
\NormalTok{  CTS}\SpecialCharTok{$}\NormalTok{Weakness }\OtherTok{\textless{}{-}} \FunctionTok{as.factor}\NormalTok{(CTS}\SpecialCharTok{$}\NormalTok{Weakness)}
\NormalTok{  predata}\OtherTok{\textless{}{-}}\NormalTok{CTS}
\NormalTok{  st\_model}\OtherTok{\textless{}{-}}\FunctionTok{preProcess}\NormalTok{(predata[,}\DecValTok{5}\SpecialCharTok{:}\DecValTok{10}\NormalTok{], }\AttributeTok{method=}\FunctionTok{c}\NormalTok{(}\StringTok{"center"}\NormalTok{,}\StringTok{"scale"}\NormalTok{))}
\NormalTok{  data}\OtherTok{\textless{}{-}}\FunctionTok{predict}\NormalTok{(st\_model, predata)}
\NormalTok{  data}\OtherTok{=}\FunctionTok{as.data.frame}\NormalTok{(data)}
\NormalTok{  ohe\_feats }\OtherTok{=} \FunctionTok{c}\NormalTok{(}\StringTok{\textquotesingle{}Sex\textquotesingle{}}\NormalTok{,}\StringTok{\textquotesingle{}Side\textquotesingle{}}\NormalTok{,}\StringTok{\textquotesingle{}Diabetes\textquotesingle{}}\NormalTok{,}\StringTok{\textquotesingle{}NP\textquotesingle{}}\NormalTok{,}\StringTok{\textquotesingle{}Weakness\textquotesingle{}}\NormalTok{)}
\NormalTok{  dummies }\OtherTok{=} \FunctionTok{dummyVars}\NormalTok{(}\SpecialCharTok{\textasciitilde{}}\NormalTok{ Sex}\SpecialCharTok{+}\NormalTok{Side}\SpecialCharTok{+}\NormalTok{Diabetes}\SpecialCharTok{+}\NormalTok{NP}\SpecialCharTok{+}\NormalTok{Weakness, }\AttributeTok{data =}\NormalTok{ data)}
\NormalTok{  df\_ohe }\OtherTok{\textless{}{-}} \FunctionTok{as.data.frame}\NormalTok{(}\FunctionTok{predict}\NormalTok{(dummies, }\AttributeTok{newdata =}\NormalTok{ data))}
\NormalTok{  df\_combined }\OtherTok{\textless{}{-}} \FunctionTok{cbind}\NormalTok{(data[,}\SpecialCharTok{{-}}\FunctionTok{c}\NormalTok{(}\FunctionTok{which}\NormalTok{(}\FunctionTok{colnames}\NormalTok{(data) }\SpecialCharTok{\%in\%}\NormalTok{ ohe\_feats))],df\_ohe)}
\NormalTok{  dat }\OtherTok{=} \FunctionTok{as.data.table}\NormalTok{(df\_combined)}
\NormalTok{  traindata}\OtherTok{\textless{}{-}}\NormalTok{dat[ind}\SpecialCharTok{==}\DecValTok{1}\NormalTok{,]}
\NormalTok{  testdata}\OtherTok{\textless{}{-}}\NormalTok{dat[ind}\SpecialCharTok{==}\DecValTok{2}\NormalTok{,]}
\NormalTok{  trainmc}\OtherTok{\textless{}{-}}\NormalTok{traindata}
\NormalTok{  testmc}\OtherTok{\textless{}{-}}\NormalTok{testdata}
\NormalTok{  trainmc}\SpecialCharTok{$}\NormalTok{Mild}\OtherTok{=}\ConstantTok{NULL}
\NormalTok{  trainmc}\SpecialCharTok{$}\NormalTok{Mod}\OtherTok{=}\ConstantTok{NULL}
\NormalTok{  trainmc}\SpecialCharTok{$}\NormalTok{Sev}\OtherTok{=}\ConstantTok{NULL}
\NormalTok{  testmc}\SpecialCharTok{$}\NormalTok{Mild}\OtherTok{=}\ConstantTok{NULL}
\NormalTok{  testmc}\SpecialCharTok{$}\NormalTok{Mod}\OtherTok{=}\ConstantTok{NULL}
\NormalTok{  testmc}\SpecialCharTok{$}\NormalTok{Sev}\OtherTok{=}\ConstantTok{NULL}
\NormalTok{  hco }\OtherTok{\textless{}{-}} \FunctionTok{nrow}\NormalTok{(CTS)}
\NormalTok{  hco }\OtherTok{\textless{}{-}}\NormalTok{ hco }\SpecialCharTok{*} \DecValTok{100}
\end{Highlighting}
\end{Shaded}
\begin{Shaded}
\begin{Highlighting}[]
\CommentTok{\# Python için Veri Önişleme }
\ImportTok{from}\NormalTok{ sklearn.preprocessing }\ImportTok{import}\NormalTok{ LabelEncoder}
\ImportTok{from}\NormalTok{ sklearn.preprocessing }\ImportTok{import}\NormalTok{ StandardScaler}
\NormalTok{LE }\OperatorTok{=}\NormalTok{ LabelEncoder().fit([}\StringTok{"mild"}\NormalTok{,}\StringTok{"moderate"}\NormalTok{,}\StringTok{"severe"}\NormalTok{])}
\NormalTok{traindata\_P }\OperatorTok{=}\NormalTok{ pd.DataFrame(r.traindata\_top)}
\NormalTok{traindata\_P.drop([}\StringTok{"Mild"}\NormalTok{,}\StringTok{"Mod"}\NormalTok{,}\StringTok{"Sev"}\NormalTok{],axis}\OperatorTok{=}\DecValTok{1}\NormalTok{,inplace}\OperatorTok{=}\VariableTok{True}\NormalTok{)}
\NormalTok{testdata\_P }\OperatorTok{=}\NormalTok{ pd.DataFrame(r.testdata\_top)}
\NormalTok{testdata\_P.drop([}\StringTok{"Mild"}\NormalTok{,}\StringTok{"Mod"}\NormalTok{,}\StringTok{"Sev"}\NormalTok{],axis}\OperatorTok{=}\DecValTok{1}\NormalTok{,inplace}\OperatorTok{=}\VariableTok{True}\NormalTok{)}
\NormalTok{X\_train, X\_test, y\_train, y\_test }\OperatorTok{=}\NormalTok{ traindata\_P.drop([}\StringTok{"Severity"}\NormalTok{],axis}\OperatorTok{=}\DecValTok{1}\NormalTok{),testdata\_P.drop([}\StringTok{"Severity"}\NormalTok{],}
\NormalTok{axis}\OperatorTok{=}\DecValTok{1}\NormalTok{),pd.DataFrame(LE.transform(traindata\_P.Severity)),}
\NormalTok{pd.DataFrame(LE.transform(testdata\_P.Severity))}
\NormalTok{Stand }\OperatorTok{=}\NormalTok{ StandardScaler().fit(r.CTS[[}\StringTok{"Age"}\NormalTok{,}\StringTok{"BMI"}\NormalTok{,}\StringTok{"CSA"}\NormalTok{,}\StringTok{"PB"}\NormalTok{,}\StringTok{"Duration"}\NormalTok{,}\StringTok{"NRS"}\NormalTok{]])}
\NormalTok{X\_train[[}\StringTok{"Age"}\NormalTok{,}\StringTok{"BMI"}\NormalTok{,}\StringTok{"CSA"}\NormalTok{,}\StringTok{"PB"}\NormalTok{,}\StringTok{"Duration"}\NormalTok{,}\StringTok{"NRS"}\NormalTok{]]}\OperatorTok{=}\NormalTok{pd.DataFrame(Stand.transform(X\_train[[}\StringTok{"Age"}\NormalTok{,}\StringTok{"BMI"}\NormalTok{,}
\StringTok{"CSA"}\NormalTok{,}\StringTok{"PB"}\NormalTok{,}\StringTok{"Duration"}\NormalTok{,}\StringTok{"NRS"}\NormalTok{]]),columns}\OperatorTok{=}\NormalTok{[}\StringTok{"Age"}\NormalTok{,}\StringTok{"BMI"}\NormalTok{,}\StringTok{"CSA"}\NormalTok{,}\StringTok{"PB"}\NormalTok{,}\StringTok{"Duration"}\NormalTok{,}\StringTok{"NRS"}\NormalTok{])}
\NormalTok{y\_train }\OperatorTok{=}\NormalTok{ y\_train.to\_numpy().ravel()}
\NormalTok{X\_test[[}\StringTok{"Age"}\NormalTok{,}\StringTok{"BMI"}\NormalTok{,}\StringTok{"CSA"}\NormalTok{,}\StringTok{"PB"}\NormalTok{,}\StringTok{"Duration"}\NormalTok{,}\StringTok{"NRS"}\NormalTok{]] }\OperatorTok{=}\NormalTok{ pd.DataFrame(Stand.transform(X\_test[[}\StringTok{"Age"}\NormalTok{,}\StringTok{"BMI"}\NormalTok{,}
\StringTok{"CSA"}\NormalTok{,}\StringTok{"PB"}\NormalTok{,}\StringTok{"Duration"}\NormalTok{,}\StringTok{"NRS"}\NormalTok{]]),columns}\OperatorTok{=}\NormalTok{[}\StringTok{"Age"}\NormalTok{,}\StringTok{"BMI"}\NormalTok{,}\StringTok{"CSA"}\NormalTok{,}\StringTok{"PB"}\NormalTok{,}\StringTok{"Duration"}\NormalTok{,}\StringTok{"NRS"}\NormalTok{])}
\NormalTok{y\_test }\OperatorTok{=}\NormalTok{ y\_test.to\_numpy().ravel()}
\end{Highlighting}
\end{Shaded}
\normalsize

\hypertarget{sux131nux131flandux131rma-uxfcuxe7-sux131nux131f}{%
\chapter{Sınıflandırma (Üç Sınıf)}\label{sux131nux131flandux131rma-uxfcuxe7-sux131nux131f}}

\hypertarget{roc-curve-grafiux11fi-ve-onevsrest-model-kurulmasux131}{%
\section{Roc Curve Grafiği ve OneVsRest Model Kurulması}\label{roc-curve-grafiux11fi-ve-onevsrest-model-kurulmasux131}}

\scriptsize
\begin{Shaded}
\begin{Highlighting}[]
\CommentTok{\# OneVsRest Modellerin ve ROC curvelerin oluşturulması}
\ImportTok{from}\NormalTok{ sklearn.metrics }\ImportTok{import}\NormalTok{ roc\_auc\_score, roc\_curve, classification\_report, confusion\_matrix}
\ImportTok{from}\NormalTok{ sklearn.metrics }\ImportTok{import}\NormalTok{ ConfusionMatrixDisplay, auc}
\ImportTok{import}\NormalTok{ pandas }\ImportTok{as}\NormalTok{ pd}
\ImportTok{from}\NormalTok{ sklearn.multiclass }\ImportTok{import}\NormalTok{ OneVsRestClassifier}
\ImportTok{import}\NormalTok{ matplotlib.pyplot }\ImportTok{as}\NormalTok{ plt}
\ImportTok{from}\NormalTok{ itertools }\ImportTok{import}\NormalTok{ cycle}
\ImportTok{from}\NormalTok{ sklearn.preprocessing }\ImportTok{import}\NormalTok{ label\_binarize}
\KeywordTok{def}\NormalTok{ roc(model):}
    \CommentTok{""" Unfitted model"""}
\NormalTok{    model\_name }\OperatorTok{=} \BuiltInTok{str}\NormalTok{(model.\_\_class\_\_).split(}\StringTok{"."}\NormalTok{)[}\OperatorTok{{-}}\DecValTok{1}\NormalTok{][:}\OperatorTok{{-}}\DecValTok{2}\NormalTok{]}
\NormalTok{    model }\OperatorTok{=}\NormalTok{ OneVsRestClassifier(model).fit(X\_train,y\_train)}
\NormalTok{    plt.figure()}
\NormalTok{    y\_pred }\OperatorTok{=}\NormalTok{ model.predict\_proba(X\_test)}
\NormalTok{    fpr }\OperatorTok{=} \BuiltInTok{dict}\NormalTok{()}
\NormalTok{    tpr }\OperatorTok{=} \BuiltInTok{dict}\NormalTok{()}
\NormalTok{    thresh }\OperatorTok{=} \BuiltInTok{dict}\NormalTok{()}
\NormalTok{    thresh\_df }\OperatorTok{=}\NormalTok{ pd.DataFrame(columns}\OperatorTok{=}\NormalTok{[}\StringTok{"Class"}\NormalTok{,}\StringTok{"Threshold"}\NormalTok{])}
\NormalTok{    roc\_auc }\OperatorTok{=} \BuiltInTok{dict}\NormalTok{()}
    \ControlFlowTok{for}\NormalTok{ i }\KeywordTok{in} \BuiltInTok{range}\NormalTok{(}\DecValTok{3}\NormalTok{):}
\NormalTok{        fpr[i], tpr[i], thresh[i] }\OperatorTok{=}\NormalTok{ roc\_curve(y\_test, y\_pred[:, i],pos\_label}\OperatorTok{=}\NormalTok{i)}
\NormalTok{        roc\_auc[i] }\OperatorTok{=}\NormalTok{ auc(fpr[i], tpr[i])}
    \ControlFlowTok{for}\NormalTok{ i,j }\KeywordTok{in} \BuiltInTok{zip}\NormalTok{(thresh.keys(),[}\StringTok{"mildVsAll"}\NormalTok{,}\StringTok{"modVsAll"}\NormalTok{,}\StringTok{"sevVsAll"}\NormalTok{]):}
\NormalTok{        thresh[j] }\OperatorTok{=}\NormalTok{ thresh.pop(i)}
        \ControlFlowTok{if}\NormalTok{ j }\OperatorTok{==} \StringTok{"sevVsAll"}\NormalTok{ : }\ControlFlowTok{break}
    \ControlFlowTok{for}\NormalTok{ i,j }\KeywordTok{in} \BuiltInTok{zip}\NormalTok{(thresh.keys(),thresh.values()):}
        \ControlFlowTok{for}\NormalTok{ x }\KeywordTok{in}\NormalTok{ j:}
\NormalTok{            thresh\_df }\OperatorTok{=}\NormalTok{ thresh\_df.append(\{}\StringTok{"Class"}\NormalTok{:i,}\StringTok{"Threshold"}\NormalTok{:x\},ignore\_index}\OperatorTok{=}\VariableTok{True}\NormalTok{)}
\NormalTok{    thresh\_df.to\_csv(}\SpecialStringTok{f\textquotesingle{}data/thresholds\_}\SpecialCharTok{\{}\NormalTok{model\_name}\SpecialCharTok{\}}\SpecialStringTok{.csv\textquotesingle{}}\NormalTok{,index}\OperatorTok{=}\VariableTok{False}\NormalTok{)}
\NormalTok{    colors }\OperatorTok{=}\NormalTok{ cycle([}\StringTok{"aqua"}\NormalTok{, }\StringTok{"darkorange"}\NormalTok{, }\StringTok{"cornflowerblue"}\NormalTok{])}
    \ControlFlowTok{for}\NormalTok{ i, color, j }\KeywordTok{in} \BuiltInTok{zip}\NormalTok{(}\BuiltInTok{range}\NormalTok{(}\DecValTok{3}\NormalTok{), colors,[}\StringTok{"Mild"}\NormalTok{,}\StringTok{"Moderate"}\NormalTok{,}\StringTok{"Severe"}\NormalTok{]):}
\NormalTok{        plt.plot(}
\NormalTok{            fpr[i],}
\NormalTok{            tpr[i],}
\NormalTok{            color}\OperatorTok{=}\NormalTok{color,}
\NormalTok{            lw}\OperatorTok{=}\DecValTok{2}\NormalTok{,}
\NormalTok{            label}\OperatorTok{=}\StringTok{"ROC curve of class }\SpecialCharTok{\{0\}}\StringTok{ (area = }\SpecialCharTok{\{1:0.2f\}}\StringTok{)"}\NormalTok{.}\BuiltInTok{format}\NormalTok{(j, roc\_auc[i])}
\NormalTok{        )}
\NormalTok{    plt.plot([}\DecValTok{0}\NormalTok{, }\DecValTok{1}\NormalTok{], [}\DecValTok{0}\NormalTok{, }\DecValTok{1}\NormalTok{], }\StringTok{"k{-}{-}"}\NormalTok{, lw}\OperatorTok{=}\DecValTok{2}\NormalTok{)}
\NormalTok{    plt.xlim([}\FloatTok{0.0}\NormalTok{, }\FloatTok{1.0}\NormalTok{])}
\NormalTok{    plt.ylim([}\FloatTok{0.0}\NormalTok{, }\FloatTok{1.05}\NormalTok{])}
\NormalTok{    plt.xlabel(}\StringTok{"False Positive Rate"}\NormalTok{)}
\NormalTok{    plt.ylabel(}\StringTok{"True Positive Rate"}\NormalTok{)}
\NormalTok{    plt.title(}\StringTok{"Some extension of Receiver operating characteristic to multiclass"}\NormalTok{)}
\NormalTok{    plt.legend(loc}\OperatorTok{=}\StringTok{"lower right"}\NormalTok{)}
\NormalTok{    plt.savefig(}\SpecialStringTok{f\textquotesingle{}figure/roc\_curve\_}\SpecialCharTok{\{}\NormalTok{model\_name}\SpecialCharTok{\}}\SpecialStringTok{.png\textquotesingle{}}\NormalTok{)}
\end{Highlighting}
\end{Shaded}
\normalsize

\hypertarget{k-en-yakux131n-komux15fuluk-modeli-kurulmasux131}{%
\section{K-En Yakın Komşuluk Modeli Kurulması}\label{k-en-yakux131n-komux15fuluk-modeli-kurulmasux131}}

\scriptsize
\begin{Shaded}
\begin{Highlighting}[]
\CommentTok{\# 3 sınıf için KNN Modeli kurma ve GridSearchCV algoritmasını hazırlama}
\ImportTok{import}\NormalTok{ numpy }\ImportTok{as}\NormalTok{ np}
\ImportTok{from}\NormalTok{ sklearn.neighbors }\ImportTok{import}\NormalTok{ KNeighborsClassifier}
\ImportTok{from}\NormalTok{ sklearn.model\_selection }\ImportTok{import}\NormalTok{ GridSearchCV}
\ImportTok{from}\NormalTok{ sklearn.metrics }\ImportTok{import}\NormalTok{ ConfusionMatrixDisplay,confusion\_matrix,classification\_report}
\ImportTok{from}\NormalTok{ sklearn.metrics }\ImportTok{import}\NormalTok{ balanced\_accuracy\_score}
\NormalTok{KNN\_model }\OperatorTok{=}\NormalTok{ KNeighborsClassifier()}
\NormalTok{params }\OperatorTok{=}\NormalTok{ \{}\StringTok{"n\_neighbors"}\NormalTok{:np.arange(}\DecValTok{5}\NormalTok{,}\DecValTok{200}\NormalTok{),}
          \StringTok{"weights"}\NormalTok{:[}\StringTok{"uniform"}\NormalTok{, }\StringTok{"distance"}\NormalTok{],}
          \StringTok{"algorithm"}\NormalTok{:[}\StringTok{"auto"}\NormalTok{,}\StringTok{"ball\_tree"}\NormalTok{,}\StringTok{"kd\_tree"}\NormalTok{,}\StringTok{"brute"}\NormalTok{]\}}
\NormalTok{GSC }\OperatorTok{=}\NormalTok{ GridSearchCV(KNN\_model,param\_grid}\OperatorTok{=}\NormalTok{params,}
\NormalTok{                   cv}\OperatorTok{=}\DecValTok{10}\NormalTok{,verbose}\OperatorTok{=}\DecValTok{1}\NormalTok{,scoring}\OperatorTok{=}\StringTok{"accuracy"}\NormalTok{).fit(X\_train,y\_train)}
\NormalTok{pd.DataFrame(GSC.cv\_results\_).to\_csv(}\StringTok{"data/KNN\_GridSearch\_Results.csv"}\NormalTok{,index}\OperatorTok{=}\VariableTok{False}\NormalTok{)}
\NormalTok{knn }\OperatorTok{=}\NormalTok{ pd.read\_csv(}\StringTok{"data/KNN\_GridSearch\_Results.csv"}\NormalTok{)}
\NormalTok{plt.figure(figsize}\OperatorTok{=}\NormalTok{(}\DecValTok{10}\NormalTok{,}\DecValTok{5}\NormalTok{),dpi}\OperatorTok{=}\DecValTok{60}\NormalTok{)}\OperatorTok{;}
\NormalTok{sns.lineplot(x}\OperatorTok{=}\NormalTok{knn.param\_n\_neighbors,y}\OperatorTok{=}\NormalTok{knn.mean\_test\_score}\OperatorTok{*}\DecValTok{100}\NormalTok{,hue}\OperatorTok{=}\NormalTok{knn.param\_weights)}\OperatorTok{;}
\NormalTok{plt.ylabel(}\StringTok{"Eğitim Verisi Doğruluk Skoru (\%)"}\NormalTok{)}\OperatorTok{;}
\NormalTok{plt.xlabel(}\StringTok{"Number of Neighbors"}\NormalTok{)}\OperatorTok{;}
\NormalTok{plt.legend(title}\OperatorTok{=}\StringTok{"Ağırlıklandırma"}\NormalTok{,loc}\OperatorTok{=}\StringTok{"upper right"}\NormalTok{,labels}\OperatorTok{=}\NormalTok{[}\StringTok{"Eşit Ağırlıklandırma"}\NormalTok{,}
\StringTok{"Uzaklığa Göre Ağırlıklandırma"}\NormalTok{])}\OperatorTok{;}
\NormalTok{plt.savefig(}\StringTok{"figure/KNN\_Grid\_Graph.png"}\NormalTok{)}\OperatorTok{;}
\end{Highlighting}
\end{Shaded}
\begin{Shaded}
\begin{Highlighting}[]
\BuiltInTok{print}\NormalTok{(}\SpecialStringTok{f\textquotesingle{}En İyi Parametreler : }\SpecialCharTok{\{}\NormalTok{GSC}\SpecialCharTok{.}\NormalTok{best\_params\_}\SpecialCharTok{\}}\SpecialStringTok{\textquotesingle{}}\NormalTok{)}
\end{Highlighting}
\end{Shaded}
\begin{Shaded}
\begin{Highlighting}[]
\CommentTok{\# En iyi parametreler ile modelin tekrar kurulması}
\NormalTok{KNN\_model }\OperatorTok{=}\NormalTok{ KNeighborsClassifier(n\_neighbors }\OperatorTok{=} \DecValTok{33}\NormalTok{,}
\NormalTok{                                 weights }\OperatorTok{=}\StringTok{\textquotesingle{}distance\textquotesingle{}}\NormalTok{).fit(X\_train,y\_train)}
\NormalTok{y\_pred }\OperatorTok{=}\NormalTok{ KNN\_model.predict(X\_test)}
\BuiltInTok{print}\NormalTok{(classification\_report(y\_test,y\_pred,target\_names}\OperatorTok{=}\NormalTok{[}\StringTok{"Mild"}\NormalTok{,}\StringTok{"Moderate"}\NormalTok{,}\StringTok{"Severe"}\NormalTok{]))}
\BuiltInTok{print}\NormalTok{(}\SpecialStringTok{f\textquotesingle{}Balanced Accuracy Score : }\SpecialCharTok{\{}\NormalTok{balanced\_accuracy\_score(y\_test,y\_pred)}\SpecialCharTok{\}}\SpecialStringTok{\textquotesingle{}}\NormalTok{)}
\end{Highlighting}
\end{Shaded}
\begin{Shaded}
\begin{Highlighting}[]
\CommentTok{\# 3 sınıflı KNN için ConfusionMatrix}
\NormalTok{ConfusionMatrixDisplay(confusion\_matrix(y\_test,y\_pred),}
\NormalTok{display\_labels}\OperatorTok{=}\NormalTok{[}\StringTok{"Mild"}\NormalTok{,}\StringTok{"Moderate"}\NormalTok{,}\StringTok{"Severe"}\NormalTok{]).plot()}\OperatorTok{;}
\NormalTok{plt.savefig(}\StringTok{"figure/knn\_conf.png"}\NormalTok{)}\OperatorTok{;}
\end{Highlighting}
\end{Shaded}
\begin{Shaded}
\begin{Highlighting}[]
\CommentTok{\# 3 sınıflı KNN için ROC curve}
\NormalTok{roc(KNeighborsClassifier(n\_neighbors }\OperatorTok{=} \DecValTok{33}\NormalTok{,}
\NormalTok{                         weights }\OperatorTok{=}\StringTok{\textquotesingle{}distance\textquotesingle{}}\NormalTok{))}
\end{Highlighting}
\end{Shaded}
\normalsize

\hypertarget{rassal-ormanlar-modeli-kurulmasux131}{%
\section{Rassal Ormanlar Modeli Kurulması}\label{rassal-ormanlar-modeli-kurulmasux131}}

\scriptsize
\begin{Shaded}
\begin{Highlighting}[]
\CommentTok{\# 3 sınıf için rassal ormanlar Modeli kurma ve GridSearchCV algoritmasını hazırlama}
\ImportTok{from}\NormalTok{ sklearn.ensemble }\ImportTok{import}\NormalTok{ RandomForestClassifier}
\ImportTok{from}\NormalTok{ sklearn.model\_selection }\ImportTok{import}\NormalTok{ GridSearchCV}
\ImportTok{from}\NormalTok{ sklearn.metrics }\ImportTok{import}\NormalTok{ ConfusionMatrixDisplay,confusion\_matrix,classification\_report}
\ImportTok{from}\NormalTok{ sklearn.metrics }\ImportTok{import}\NormalTok{ balanced\_accuracy\_score}
\NormalTok{RFC\_model }\OperatorTok{=}\NormalTok{ RandomForestClassifier()}
\NormalTok{param\_grid }\OperatorTok{=}\NormalTok{ \{}\StringTok{"n\_estimators"}\NormalTok{:np.arange(}\DecValTok{350}\NormalTok{,}\DecValTok{1000}\NormalTok{,}\DecValTok{50}\NormalTok{),}
              \StringTok{"criterion"}\NormalTok{:[}\StringTok{"gini"}\NormalTok{,}\StringTok{"index"}\NormalTok{],}
              \StringTok{"max\_features"}\NormalTok{:[}\StringTok{"auto"}\NormalTok{,}\StringTok{"sqrt"}\NormalTok{,}\StringTok{"log2"}\NormalTok{],}
              \StringTok{"ccp\_alpha"}\NormalTok{:[}\FloatTok{0.01}\NormalTok{,}\FloatTok{0.05}\NormalTok{,}\FloatTok{.1}\NormalTok{,}\FloatTok{0.3}\NormalTok{,}\FloatTok{.5}\NormalTok{,}\FloatTok{.7}\NormalTok{,}\FloatTok{.9}\NormalTok{,}\DecValTok{1}\NormalTok{],}
              \StringTok{"max\_samples"}\NormalTok{:np.arange(}\DecValTok{1}\NormalTok{,X\_train.shape[}\DecValTok{1}\NormalTok{],}\DecValTok{1}\NormalTok{)\}}
\NormalTok{GSC }\OperatorTok{=}\NormalTok{ GridSearchCV(RFC\_model,param\_grid}\OperatorTok{=}\NormalTok{params,}
\NormalTok{                   cv}\OperatorTok{=}\DecValTok{10}\NormalTok{,verbose}\OperatorTok{=}\DecValTok{1}\NormalTok{,scoring}\OperatorTok{=}\StringTok{"accuracy"}\NormalTok{,random\_state}\OperatorTok{=}\DecValTok{13}\NormalTok{).fit(X\_train,y\_train)}
\NormalTok{results }\OperatorTok{=}\NormalTok{ pd.read\_csv(}\StringTok{"data/RF\_GridSearch\_Results.csv"}\NormalTok{)}
\NormalTok{ginis }\OperatorTok{=}\NormalTok{ results[results[}\StringTok{"param\_criterion"}\NormalTok{] }\OperatorTok{==} \StringTok{"gini"}\NormalTok{]}
\NormalTok{ginis }\OperatorTok{=}\NormalTok{ ginis[ginis[}\StringTok{"param\_ccp\_alpha"}\NormalTok{] }\OperatorTok{\textless{}=} \FloatTok{0.1}\NormalTok{]}
\NormalTok{ginis }\OperatorTok{=}\NormalTok{ ginis[ginis[}\StringTok{"mean\_test\_score"}\NormalTok{]}\OperatorTok{\textgreater{}=}\FloatTok{0.704}\NormalTok{]}
\CommentTok{\#Grafik Çizimi}
\NormalTok{plt.figure(figsize}\OperatorTok{=}\NormalTok{(}\DecValTok{10}\NormalTok{,}\DecValTok{5}\NormalTok{),dpi}\OperatorTok{=}\DecValTok{60}\NormalTok{)}\OperatorTok{;}
\NormalTok{plt.ylim(}\FloatTok{0.703}\OperatorTok{*}\DecValTok{100}\NormalTok{,}\BuiltInTok{round}\NormalTok{(ginis.mean\_test\_score.unique().}\BuiltInTok{max}\NormalTok{()}\OperatorTok{*}\DecValTok{100}\NormalTok{,}\DecValTok{2}\NormalTok{)}\OperatorTok{+}\FloatTok{0.1}\NormalTok{)}\OperatorTok{;}
\NormalTok{sns.lineplot(y}\OperatorTok{=}\NormalTok{ginis.mean\_test\_score}\OperatorTok{*}\DecValTok{100}\NormalTok{,x}\OperatorTok{=}\NormalTok{ginis.param\_n\_estimators,hue}\OperatorTok{=}\NormalTok{ginis.param\_ccp\_alpha,}
\NormalTok{palette}\OperatorTok{=}\NormalTok{sns.color\_palette(n\_colors}\OperatorTok{=}\DecValTok{3}\NormalTok{),err\_style}\OperatorTok{=}\VariableTok{None}\NormalTok{)}\OperatorTok{;}
\NormalTok{plt.ylabel(}\StringTok{"Eğitim Verisi Doğruluk Skoru (\%)"}\NormalTok{)}\OperatorTok{;}
\NormalTok{plt.xlabel(}\StringTok{"Number of Trees in Forest"}\NormalTok{)}\OperatorTok{;}
\NormalTok{plt.legend(title}\OperatorTok{=}\StringTok{"Learning Rate"}\NormalTok{,loc}\OperatorTok{=}\StringTok{"upper right"}\NormalTok{,labels}\OperatorTok{=}\NormalTok{[}\FloatTok{0.01}\NormalTok{,}\FloatTok{0.05}\NormalTok{,}\FloatTok{0.1}\NormalTok{])}\OperatorTok{;}
\NormalTok{plt.savefig(}\StringTok{"figure/RF\_Grid\_Graph.png"}\NormalTok{)}\OperatorTok{;}
\end{Highlighting}
\end{Shaded}
\begin{Shaded}
\begin{Highlighting}[]
\BuiltInTok{print}\NormalTok{(}\SpecialStringTok{f\textquotesingle{}En İyi Parametreler : }\SpecialCharTok{\{}\NormalTok{GSC}\SpecialCharTok{.}\NormalTok{best\_params\_}\SpecialCharTok{\}}\SpecialStringTok{\textquotesingle{}}\NormalTok{)}
\end{Highlighting}
\end{Shaded}
\begin{Shaded}
\begin{Highlighting}[]
\NormalTok{RFC\_model }\OperatorTok{=}\NormalTok{ RandomForestClassifier(ccp\_alpha}\OperatorTok{=}\FloatTok{0.05}\NormalTok{,criterion}\OperatorTok{=}\StringTok{"gini"}\NormalTok{,max\_features}\OperatorTok{=}\StringTok{"auto"}\NormalTok{,}
\NormalTok{                                max\_samples}\OperatorTok{=}\DecValTok{10}\NormalTok{,n\_estimators}\OperatorTok{=}\DecValTok{350}\NormalTok{,random\_state}\OperatorTok{=}\DecValTok{13}\NormalTok{).fit(X\_train,y\_train)}
\NormalTok{y\_pred }\OperatorTok{=}\NormalTok{ RFC\_model.predict(X\_test)}
\BuiltInTok{print}\NormalTok{(classification\_report(y\_test,y\_pred,target\_names}\OperatorTok{=}\NormalTok{[}\StringTok{"Mild"}\NormalTok{,}\StringTok{"Moderate"}\NormalTok{,}\StringTok{"Severe"}\NormalTok{]))}
\BuiltInTok{print}\NormalTok{(}\SpecialStringTok{f\textquotesingle{}Balanced Accuracy Score : }\SpecialCharTok{\{}\NormalTok{balanced\_accuracy\_score(y\_test,y\_pred)}\SpecialCharTok{\}}\SpecialStringTok{\textquotesingle{}}\NormalTok{)}
\end{Highlighting}
\end{Shaded}
\begin{Shaded}
\begin{Highlighting}[]
\NormalTok{ConfusionMatrixDisplay(confusion\_matrix(y\_test,y\_pred),}
\NormalTok{display\_labels}\OperatorTok{=}\NormalTok{[}\StringTok{"Mild"}\NormalTok{,}\StringTok{"Moderate"}\NormalTok{,}\StringTok{"Severe"}\NormalTok{]).plot()}\OperatorTok{;}
\NormalTok{plt.savefig(}\StringTok{"figure/rfc\_conf.png"}\NormalTok{)}
\end{Highlighting}
\end{Shaded}
\begin{Shaded}
\begin{Highlighting}[]
\NormalTok{roc(RandomForestClassifier(ccp\_alpha}\OperatorTok{=}\FloatTok{0.01}\NormalTok{,criterion}\OperatorTok{=}\StringTok{"gini"}\NormalTok{,max\_features}\OperatorTok{=}\StringTok{"sqrt"}\NormalTok{,}
\NormalTok{                                   max\_samples}\OperatorTok{=}\DecValTok{10}\NormalTok{,n\_estimators}\OperatorTok{=}\DecValTok{900}\NormalTok{,random\_state}\OperatorTok{=}\DecValTok{13}\NormalTok{))}
\end{Highlighting}
\end{Shaded}
\normalsize

\hypertarget{xgboost-modeli-kurulmasux131}{%
\section{XGBoost Modeli Kurulması}\label{xgboost-modeli-kurulmasux131}}

\scriptsize
\begin{Shaded}
\begin{Highlighting}[]
\CommentTok{\# 3 sınıf için XGB Modeli}
\ImportTok{from}\NormalTok{ xgboost.sklearn }\ImportTok{import}\NormalTok{ XGBClassifier}
\ImportTok{from}\NormalTok{ sklearn.model\_selection }\ImportTok{import}\NormalTok{ GridSearchCV}
\ImportTok{from}\NormalTok{ sklearn.metrics }\ImportTok{import}\NormalTok{ ConfusionMatrixDisplay,confusion\_matrix,classification\_report}
\ImportTok{from}\NormalTok{ sklearn.metrics }\ImportTok{import}\NormalTok{ balanced\_accuracy\_score}
\NormalTok{XGB\_model }\OperatorTok{=}\NormalTok{ XGBClassifier()}
\NormalTok{param\_grid }\OperatorTok{=}\NormalTok{ \{}\StringTok{"booster"}\NormalTok{:[}\StringTok{"gbtree"}\NormalTok{,}\StringTok{"gblinear"}\NormalTok{],}
              \StringTok{"eta"}\NormalTok{:np.arange(}\DecValTok{0}\NormalTok{,}\DecValTok{1}\NormalTok{,}\FloatTok{0.1}\NormalTok{),}
              \StringTok{"min\_child\_weight"}\NormalTok{:np.arange(}\DecValTok{0}\NormalTok{,}\DecValTok{5}\NormalTok{,}\DecValTok{1}\NormalTok{),}
              \StringTok{"max\_depth"}\NormalTok{:np.arange(}\DecValTok{3}\NormalTok{,}\DecValTok{10}\NormalTok{,}\DecValTok{1}\NormalTok{),}
              \StringTok{"gamma"}\NormalTok{:np.arange(}\DecValTok{0}\NormalTok{,}\DecValTok{1}\NormalTok{,}\FloatTok{0.1}\NormalTok{),}
              \StringTok{"sumsample"}\NormalTok{:np.arange(}\DecValTok{0}\NormalTok{,}\DecValTok{1}\NormalTok{,}\FloatTok{0.1}\NormalTok{),}
              \StringTok{"colsample\_bytree"}\NormalTok{:np.arange(}\DecValTok{0}\NormalTok{,}\DecValTok{1}\NormalTok{,}\FloatTok{.1}\NormalTok{),}
              \StringTok{"n\_estimators"}\NormalTok{:np.arange(}\DecValTok{0}\NormalTok{,}\DecValTok{1000}\NormalTok{,}\DecValTok{50}\NormalTok{),}
              \StringTok{"objective"}\NormalTok{:[}\StringTok{"multi:softmax"}\NormalTok{,}\StringTok{"multi:softprob"}\NormalTok{],}
              \StringTok{"eval\_metric"}\NormalTok{:[}\StringTok{"auc"}\NormalTok{],}
              \StringTok{"use\_label\_encoder"}\NormalTok{:[}\VariableTok{False}\NormalTok{]\}}
\NormalTok{GSC }\OperatorTok{=}\NormalTok{ GridSearchCV(XGB\_model,param\_grid}\OperatorTok{=}\NormalTok{param\_grid,cv}\OperatorTok{=}\DecValTok{10}\NormalTok{,verbose}\OperatorTok{=}\DecValTok{1}\NormalTok{,n\_jobs}\OperatorTok{={-}}\DecValTok{1}\NormalTok{,scoring}\OperatorTok{=}\StringTok{"accuracy"}\NormalTok{)}
\NormalTok{GSC.fit(X\_train,y\_train)}
\NormalTok{resultsxgb }\OperatorTok{=}\NormalTok{ pd.read\_csv(}\StringTok{"data/XGBoost\_GridSearch\_Results.csv"}\NormalTok{,sep}\OperatorTok{=}\StringTok{";"}\NormalTok{)}
\NormalTok{resultsxgb }\OperatorTok{=}\NormalTok{ resultsxgb.sort\_values(}\StringTok{"rank\_test\_score"}\NormalTok{)}
\NormalTok{sorted\_xgb }\OperatorTok{=}\NormalTok{ resultsxgb[[}\StringTok{"param\_booster"}\NormalTok{,}\StringTok{"param\_eta"}\NormalTok{,}\StringTok{"param\_max\_depth"}\NormalTok{,}\StringTok{"param\_min\_child\_weight"}\NormalTok{,}
\StringTok{"param\_n\_estimators"}\NormalTok{,}\StringTok{"param\_objective"}\NormalTok{,}\StringTok{"mean\_test\_score"}\NormalTok{,}\StringTok{"rank\_test\_score"}\NormalTok{]]}
\CommentTok{\#Grafik Çizimi}
\NormalTok{plt.figure(figsize}\OperatorTok{=}\NormalTok{(}\DecValTok{15}\NormalTok{,}\DecValTok{10}\NormalTok{),dpi}\OperatorTok{=}\DecValTok{100}\NormalTok{)}\OperatorTok{;}
\NormalTok{plt.ylim(}\FloatTok{0.72}\OperatorTok{*}\DecValTok{100}\NormalTok{,}\BuiltInTok{round}\NormalTok{(sorted\_xgb.mean\_test\_score.unique().}\BuiltInTok{max}\NormalTok{()}\OperatorTok{*}\DecValTok{100}\NormalTok{,}\DecValTok{2}\NormalTok{)}\OperatorTok{+}\FloatTok{0.05}\NormalTok{)}\OperatorTok{;}
\ControlFlowTok{for}\NormalTok{ i }\KeywordTok{in}\NormalTok{ sorted\_xgb.param\_eta.unique():}
    \ControlFlowTok{if}\NormalTok{ i }\OperatorTok{\textless{}=}\FloatTok{0.4}\NormalTok{:}
\NormalTok{        x }\OperatorTok{=}\NormalTok{ sorted\_xgb[sorted\_xgb.param\_eta }\OperatorTok{==}\NormalTok{ i].groupby(}\StringTok{"param\_n\_estimators"}\NormalTok{).}\BuiltInTok{max}\NormalTok{()}
\NormalTok{        sns.lineplot(x}\OperatorTok{=}\NormalTok{x.index,y}\OperatorTok{=}\NormalTok{x.mean\_test\_score}\OperatorTok{*}\DecValTok{100}\NormalTok{)}\OperatorTok{;}
\NormalTok{plt.ylabel(}\StringTok{"Eğitim Verisi Doğruluk Skoru (\%)"}\NormalTok{)}\OperatorTok{;}
\NormalTok{plt.legend([}\FloatTok{.1}\NormalTok{,}\FloatTok{.4}\NormalTok{,}\FloatTok{.2}\NormalTok{,}\FloatTok{.3}\NormalTok{],loc}\OperatorTok{=}\DecValTok{1}\NormalTok{,title}\OperatorTok{=}\StringTok{"Learning Rate"}\NormalTok{)}\OperatorTok{;}
\NormalTok{plt.savefig(}\StringTok{"figure/XGB\_Grid\_Graph.png"}\NormalTok{)}\OperatorTok{;}
\end{Highlighting}
\end{Shaded}
\begin{Shaded}
\begin{Highlighting}[]
\BuiltInTok{print}\NormalTok{(}\SpecialStringTok{f\textquotesingle{}En İyi Parametreler : }\SpecialCharTok{\{}\NormalTok{GSC}\SpecialCharTok{.}\NormalTok{best\_params\_}\SpecialCharTok{\}}\SpecialStringTok{\textquotesingle{}}\NormalTok{)}
\end{Highlighting}
\end{Shaded}
\begin{Shaded}
\begin{Highlighting}[]
\NormalTok{XGB\_model }\OperatorTok{=}\NormalTok{ XGBClassifier(booster}\OperatorTok{=}\StringTok{"gbtree"}\NormalTok{,eta}\OperatorTok{=}\StringTok{"0.1"}\NormalTok{,max\_depth}\OperatorTok{=}\DecValTok{3}\NormalTok{,min\_child\_weight}\OperatorTok{=}\DecValTok{10}\NormalTok{,n\_estimators}\OperatorTok{=}\DecValTok{100}\NormalTok{,}
\NormalTok{objective}\OperatorTok{=}\StringTok{"multi:softprob"}\NormalTok{,eval\_metric}\OperatorTok{=}\StringTok{"auc"}\NormalTok{,use\_label\_encoder}\OperatorTok{=}\VariableTok{False}\NormalTok{,num\_class}\OperatorTok{=}\DecValTok{2}\NormalTok{).fit(X\_train,y\_train)}
\NormalTok{y\_pred }\OperatorTok{=}\NormalTok{ XGB\_model.predict(X\_test)}
\BuiltInTok{print}\NormalTok{(classification\_report(y\_test,y\_pred,target\_names}\OperatorTok{=}\NormalTok{[}\StringTok{"Mild"}\NormalTok{,}\StringTok{"Moderate"}\NormalTok{,}\StringTok{"Severe"}\NormalTok{]))}
\BuiltInTok{print}\NormalTok{(}\SpecialStringTok{f\textquotesingle{}Balanced Accuracy Score : }\SpecialCharTok{\{}\NormalTok{balanced\_accuracy\_score(y\_test,y\_pred)}\SpecialCharTok{\}}\SpecialStringTok{\textquotesingle{}}\NormalTok{)}
\end{Highlighting}
\end{Shaded}
\begin{Shaded}
\begin{Highlighting}[]
\NormalTok{ConfusionMatrixDisplay(confusion\_matrix(y\_test,y\_pred),}
\NormalTok{                       display\_labels}\OperatorTok{=}\NormalTok{[}\StringTok{"Mild"}\NormalTok{,}\StringTok{"Moderate"}\NormalTok{,}\StringTok{"Severe"}\NormalTok{]).plot()}\OperatorTok{;}
\NormalTok{plt.savefig(}\StringTok{"figure/xgb\_conf.png"}\NormalTok{)}\OperatorTok{;}
\end{Highlighting}
\end{Shaded}
\begin{Shaded}
\begin{Highlighting}[]
\NormalTok{roc(XGBClassifier(booster}\OperatorTok{=}\StringTok{"gbtree"}\NormalTok{,eta}\OperatorTok{=}\StringTok{"0.1"}\NormalTok{,max\_depth}\OperatorTok{=}\DecValTok{3}\NormalTok{,min\_child\_weight}\OperatorTok{=}\DecValTok{10}\NormalTok{,n\_estimators}\OperatorTok{=}\DecValTok{100}\NormalTok{,}
\NormalTok{objective}\OperatorTok{=}\StringTok{"multi:softprob"}\NormalTok{,eval\_metric}\OperatorTok{=}\StringTok{"auc"}\NormalTok{,use\_label\_encoder}\OperatorTok{=}\VariableTok{False}\NormalTok{,num\_class}\OperatorTok{=}\DecValTok{2}\NormalTok{))}
\end{Highlighting}
\end{Shaded}
\normalsize

\hypertarget{yapay-sinir-aux11flarux131-modeli-kurulmasux131}{%
\section{Yapay Sinir Ağları Modeli Kurulması}\label{yapay-sinir-aux11flarux131-modeli-kurulmasux131}}

\scriptsize
\begin{Shaded}
\begin{Highlighting}[]
\CommentTok{\# 3 sınıflı sinir ağları modeli}
\ImportTok{from}\NormalTok{ sklearn.neural\_network }\ImportTok{import}\NormalTok{ MLPClassifier}
\ImportTok{from}\NormalTok{ sklearn.model\_selection }\ImportTok{import}\NormalTok{ GridSearchCV}
\ImportTok{from}\NormalTok{ sklearn.metrics }\ImportTok{import}\NormalTok{ ConfusionMatrixDisplay,confusion\_matrix,classification\_report}
\ImportTok{from}\NormalTok{ sklearn.metrics }\ImportTok{import}\NormalTok{ balanced\_accuracy\_score}
\NormalTok{params }\OperatorTok{=}\NormalTok{ \{}\StringTok{"hidden\_layer\_sizes"}\NormalTok{:np.arange(}\DecValTok{1}\NormalTok{,}\DecValTok{26}\NormalTok{,}\DecValTok{1}\NormalTok{),}
          \StringTok{"learning\_rate"}\NormalTok{:[}\StringTok{"adaptive"}\NormalTok{,}\StringTok{"constant"}\NormalTok{,}\StringTok{"invscaling"}\NormalTok{],}
          \StringTok{"activation"}\NormalTok{:[}\StringTok{"identity"}\NormalTok{,}\StringTok{"logistic"}\NormalTok{,}\StringTok{"tanh"}\NormalTok{,}\StringTok{"relu"}\NormalTok{]\}}
\NormalTok{gridd }\OperatorTok{=}\NormalTok{ GridSearchCV(MLPClassifier(random\_state}\OperatorTok{=}\DecValTok{13}\NormalTok{),param\_grid}\OperatorTok{=}\NormalTok{params,cv}\OperatorTok{=}\DecValTok{10}\NormalTok{,}
\NormalTok{                                   verbose}\OperatorTok{=}\DecValTok{1}\NormalTok{,n\_jobs}\OperatorTok{={-}}\DecValTok{1}\NormalTok{,scoring}\OperatorTok{=}\StringTok{"accuracy"}\NormalTok{)}
\NormalTok{gridd.fit(X\_train,y\_train)                                  }
\NormalTok{pd.DataFrame(gridd.cv\_results\_).to\_csv(}\StringTok{"NN\_GridSearch\_Results.csv"}\NormalTok{,sep}\OperatorTok{=}\StringTok{";"}\NormalTok{,index}\OperatorTok{=}\VariableTok{False}\NormalTok{)}
\NormalTok{result\_nn }\OperatorTok{=}\NormalTok{ pd.read\_csv(}\StringTok{"data/NN\_GridSearch\_Results.csv"}\NormalTok{,sep}\OperatorTok{=}\StringTok{";"}\NormalTok{)}
\CommentTok{\#Grafik Çizimi}
\NormalTok{plt.figure(figsize}\OperatorTok{=}\NormalTok{(}\DecValTok{15}\NormalTok{,}\DecValTok{10}\NormalTok{),dpi}\OperatorTok{=}\DecValTok{100}\NormalTok{)}\OperatorTok{;}
\NormalTok{sns.lineplot(x}\OperatorTok{=}\NormalTok{result\_nn[}\StringTok{"param\_hidden\_layer\_sizes"}\NormalTok{],y}\OperatorTok{=}\NormalTok{result\_nn[}\StringTok{"mean\_test\_score"}\NormalTok{]}\OperatorTok{*}\DecValTok{100}\NormalTok{,}
\NormalTok{             hue}\OperatorTok{=}\NormalTok{result\_nn[}\StringTok{"param\_activation"}\NormalTok{])}\OperatorTok{;}
\NormalTok{plt.xlabel(}\StringTok{"Hidden Layer Sizes"}\NormalTok{)}\OperatorTok{;}
\NormalTok{plt.ylabel(}\StringTok{"Eğitim Verisi Doğruluk Skoru (\%)"}\NormalTok{)}\OperatorTok{;}
\NormalTok{plt.legend(title}\OperatorTok{=}\StringTok{"Aktivasyon Fonksiyonu"}\NormalTok{)}\OperatorTok{;}
\NormalTok{plt.savefig(}\StringTok{"figure/NN\_Grid\_Graph.png"}\NormalTok{)}\OperatorTok{;}
\end{Highlighting}
\end{Shaded}
\begin{Shaded}
\begin{Highlighting}[]
\BuiltInTok{print}\NormalTok{(}\SpecialStringTok{f\textquotesingle{}En İyi Parametreler : }\SpecialCharTok{\{}\NormalTok{gridd}\SpecialCharTok{.}\NormalTok{best\_params\_}\SpecialCharTok{\}}\SpecialStringTok{\textquotesingle{}}\NormalTok{)}
\end{Highlighting}
\end{Shaded}
\begin{Shaded}
\begin{Highlighting}[]
\NormalTok{NN\_model }\OperatorTok{=}\NormalTok{ MLPClassifier(activation}\OperatorTok{=}\StringTok{"relu"}\NormalTok{,hidden\_layer\_sizes}\OperatorTok{=}\DecValTok{19}\NormalTok{,learning\_rate}\OperatorTok{=}\StringTok{"adaptive"}\NormalTok{,}
\NormalTok{                         random\_state}\OperatorTok{=}\DecValTok{13}\NormalTok{,max\_iter}\OperatorTok{=}\DecValTok{3000}\NormalTok{).fit(X\_train,y\_train)}
\NormalTok{y\_pred }\OperatorTok{=}\NormalTok{ NN\_model.predict(X\_test)}
\BuiltInTok{print}\NormalTok{(classification\_report(y\_test,y\_pred,target\_names}\OperatorTok{=}\NormalTok{[}\StringTok{"Mild"}\NormalTok{,}\StringTok{"Moderate"}\NormalTok{,}\StringTok{"Severe"}\NormalTok{]))}
\BuiltInTok{print}\NormalTok{(}\SpecialStringTok{f\textquotesingle{}Balanced Accuracy Score : }\SpecialCharTok{\{}\NormalTok{balanced\_accuracy\_score(y\_test,y\_pred)}\SpecialCharTok{\}}\SpecialStringTok{\textquotesingle{}}\NormalTok{)}
\end{Highlighting}
\end{Shaded}
\begin{Shaded}
\begin{Highlighting}[]
\NormalTok{ConfusionMatrixDisplay(confusion\_matrix(y\_test,y\_pred),}
\NormalTok{                       display\_labels}\OperatorTok{=}\NormalTok{[}\StringTok{"Mild"}\NormalTok{,}\StringTok{"Moderate"}\NormalTok{,}\StringTok{"Severe"}\NormalTok{]).plot()}\OperatorTok{;}
\NormalTok{plt.savefig(}\StringTok{"figure/nn\_conf.png"}\NormalTok{)}
\end{Highlighting}
\end{Shaded}
\begin{Shaded}
\begin{Highlighting}[]
\NormalTok{roc(MLPClassifier(activation}\OperatorTok{=}\StringTok{"relu"}\NormalTok{,hidden\_layer\_sizes}\OperatorTok{=}\DecValTok{19}\NormalTok{,learning\_rate}\OperatorTok{=}\StringTok{"adaptive"}\NormalTok{,}
\NormalTok{                  random\_state}\OperatorTok{=}\DecValTok{13}\NormalTok{,max\_iter}\OperatorTok{=}\DecValTok{3000}\NormalTok{))}
\end{Highlighting}
\end{Shaded}
\normalsize

\hypertarget{sux131nux131flandux131rma-iki-sux131nux131f}{%
\chapter{Sınıflandırma (İki Sınıf)}\label{sux131nux131flandux131rma-iki-sux131nux131f}}

\hypertarget{iki-sux131nux131flux131-iuxe7in-veri-uxf6niux15fleme}{%
\section{İki Sınıflı İçin Veri Önişleme}\label{iki-sux131nux131flux131-iuxe7in-veri-uxf6niux15fleme}}

\scriptsize
\begin{Shaded}
\begin{Highlighting}[]
\CommentTok{\# 3 sınıfı 2 sınıfa indirgeme}
\ImportTok{from}\NormalTok{ sklearn.model\_selection }\ImportTok{import}\NormalTok{ train\_test\_split}
\NormalTok{bin\_data }\OperatorTok{=}\NormalTok{ pd.read\_csv(}\StringTok{"data/binary\_data.csv"}\NormalTok{,sep}\OperatorTok{=}\StringTok{";"}\NormalTok{)}
\NormalTok{X\_bin }\OperatorTok{=}\NormalTok{ bin\_data.drop([}\StringTok{"Severity"}\NormalTok{,}\StringTok{"New\_Sev"}\NormalTok{],axis}\OperatorTok{=}\DecValTok{1}\NormalTok{)}
\NormalTok{y\_bin }\OperatorTok{=}\NormalTok{ bin\_data[}\StringTok{"New\_Sev"}\NormalTok{]}
\NormalTok{Xbin\_train, Xbin\_test, ybin\_train, ybin\_test }\OperatorTok{=}\NormalTok{ train\_test\_split(X\_bin,y\_bin,test\_size}\OperatorTok{=}\FloatTok{0.3}\NormalTok{,}
\NormalTok{                                                                stratify}\OperatorTok{=}\NormalTok{y\_bin,random\_state}\OperatorTok{=}\DecValTok{13}\NormalTok{)}
\KeywordTok{def}\NormalTok{ roc\_bin(model):}
\NormalTok{    model\_name }\OperatorTok{=} \BuiltInTok{str}\NormalTok{(model.\_\_class\_\_).split(}\StringTok{"."}\NormalTok{)[}\OperatorTok{{-}}\DecValTok{1}\NormalTok{][:}\OperatorTok{{-}}\DecValTok{2}\NormalTok{]}
\NormalTok{    ybin\_pred2 }\OperatorTok{=}\NormalTok{ model.predict\_proba(Xbin\_test)[::, }\DecValTok{1}\NormalTok{]}
\NormalTok{    plt.figure(figsize}\OperatorTok{=}\NormalTok{(}\DecValTok{10}\NormalTok{,}\DecValTok{5}\NormalTok{),dpi}\OperatorTok{=}\DecValTok{100}\NormalTok{)}
\NormalTok{    fpr, tpr, thresh }\OperatorTok{=}\NormalTok{ roc\_curve(ybin\_test, ybin\_pred2)}
\NormalTok{    roc\_auc }\OperatorTok{=}\NormalTok{ auc(fpr, tpr)}
\NormalTok{    plt.plot(}
\NormalTok{        fpr,}
\NormalTok{        tpr,}
\NormalTok{        lw}\OperatorTok{=}\DecValTok{2}\NormalTok{,}
\NormalTok{        label}\OperatorTok{=}\SpecialStringTok{f"AUC = }\SpecialCharTok{\{}\NormalTok{roc\_auc}\SpecialCharTok{\}}\SpecialStringTok{"}
\NormalTok{        )}
\NormalTok{    plt.plot([}\DecValTok{0}\NormalTok{, }\DecValTok{1}\NormalTok{], [}\DecValTok{0}\NormalTok{, }\DecValTok{1}\NormalTok{], }\StringTok{"k{-}{-}"}\NormalTok{, lw}\OperatorTok{=}\DecValTok{2}\NormalTok{)}
\NormalTok{    plt.xlim([}\FloatTok{0.0}\NormalTok{, }\FloatTok{1.0}\NormalTok{])}
\NormalTok{    plt.ylim([}\FloatTok{0.0}\NormalTok{, }\FloatTok{1.05}\NormalTok{])}
\NormalTok{    plt.xlabel(}\StringTok{"Yanlış Pozitif Oranı"}\NormalTok{)}
\NormalTok{    plt.ylabel(}\StringTok{"Doğru Pozitif Oranı"}\NormalTok{)}
\NormalTok{    plt.title(}\StringTok{"ROC Eğrisi"}\NormalTok{)}
\NormalTok{    plt.legend(loc}\OperatorTok{=}\StringTok{"lower right"}\NormalTok{)}
\NormalTok{    plt.savefig(}\SpecialStringTok{f\textquotesingle{}figure/}\SpecialCharTok{\{}\NormalTok{model\_name}\SpecialCharTok{\}}\SpecialStringTok{\_binary\_roc.png\textquotesingle{}}\NormalTok{)}
\end{Highlighting}
\end{Shaded}
\normalsize

\hypertarget{k-en-yakux131n-komux15fuluk-modeli-kurulmasux131-1}{%
\section{K-En Yakın Komşuluk Modeli Kurulması}\label{k-en-yakux131n-komux15fuluk-modeli-kurulmasux131-1}}

\scriptsize
\begin{Shaded}
\begin{Highlighting}[]
\NormalTok{KNN\_model\_bin }\OperatorTok{=}\NormalTok{ KNeighborsClassifier()}
\NormalTok{params }\OperatorTok{=}\NormalTok{ \{}\StringTok{"n\_neighbors"}\NormalTok{:np.arange(}\DecValTok{5}\NormalTok{,}\DecValTok{200}\NormalTok{),}
          \StringTok{"weights"}\NormalTok{:[}\StringTok{"uniform"}\NormalTok{, }\StringTok{"distance"}\NormalTok{],}
          \StringTok{"algorithm"}\NormalTok{:[}\StringTok{"auto"}\NormalTok{,}\StringTok{"ball\_tree"}\NormalTok{,}\StringTok{"kd\_tree"}\NormalTok{,}\StringTok{"brute"}\NormalTok{]\}}
\NormalTok{GSC }\OperatorTok{=}\NormalTok{ GridSearchCV(KNN\_model\_bin,param\_grid}\OperatorTok{=}\NormalTok{params,cv}\OperatorTok{=}\DecValTok{10}\NormalTok{,verbose}\OperatorTok{=}\DecValTok{1}\NormalTok{,scoring}\OperatorTok{=}\StringTok{"accuracy"}\NormalTok{,n\_jobs}\OperatorTok{={-}}\DecValTok{1}\NormalTok{)}
\NormalTok{GSC.fit(Xbin\_train,ybin\_train)}
\NormalTok{pd.DataFrame(GSC.cv\_results\_).to\_csv(}\StringTok{"data/KNN\_bin\_GridSearch\_Results.csv"}\NormalTok{,index}\OperatorTok{=}\VariableTok{False}\NormalTok{)}
\NormalTok{knn\_bin }\OperatorTok{=}\NormalTok{ pd.read\_csv(}\StringTok{"data/KNN\_bin\_GridSearch\_Results.csv"}\NormalTok{,sep}\OperatorTok{=}\StringTok{";"}\NormalTok{)}
\NormalTok{plt.figure(figsize}\OperatorTok{=}\NormalTok{(}\DecValTok{10}\NormalTok{,}\DecValTok{5}\NormalTok{),dpi}\OperatorTok{=}\DecValTok{60}\NormalTok{)}\OperatorTok{;}
\NormalTok{sns.lineplot(x}\OperatorTok{=}\NormalTok{knn\_bin[}\StringTok{"param\_n\_neighbors"}\NormalTok{],y}\OperatorTok{=}\NormalTok{knn\_bin[}\StringTok{"mean\_test\_score"}\NormalTok{]}\OperatorTok{*}\DecValTok{100}\NormalTok{,}
\NormalTok{             hue}\OperatorTok{=}\NormalTok{knn\_bin[}\StringTok{"param\_weights"}\NormalTok{])}\OperatorTok{;}
\NormalTok{plt.ylabel(}\StringTok{"Eğitim Verisi Doğruluk Skoru (\%)"}\NormalTok{)}\OperatorTok{;}
\NormalTok{plt.xlabel(}\StringTok{"Number of Neighbors"}\NormalTok{)}\OperatorTok{;}
\NormalTok{plt.legend(title}\OperatorTok{=}\StringTok{"Ağırlıklandırma"}\NormalTok{,loc}\OperatorTok{=}\StringTok{"upper right"}\NormalTok{,}
\NormalTok{           labels}\OperatorTok{=}\NormalTok{[}\StringTok{"Eşit Ağırlıklandırma"}\NormalTok{,}\StringTok{"Uzaklığa Göre Ağırlıklandırma"}\NormalTok{])}\OperatorTok{;}
\NormalTok{plt.savefig(}\StringTok{"figure/KNN\_bin\_Grid\_Graph.png"}\NormalTok{)}\OperatorTok{;}
\end{Highlighting}
\end{Shaded}
\begin{Shaded}
\begin{Highlighting}[]
\BuiltInTok{print}\NormalTok{(}\SpecialStringTok{f\textquotesingle{}En İyi Parametreler : }\SpecialCharTok{\{}\NormalTok{GSC}\SpecialCharTok{.}\NormalTok{best\_params\_}\SpecialCharTok{\}}\SpecialStringTok{\textquotesingle{}}\NormalTok{)}
\end{Highlighting}
\end{Shaded}
\begin{Shaded}
\begin{Highlighting}[]
\NormalTok{KNN\_modelbin }\OperatorTok{=}\NormalTok{ KNeighborsClassifier(n\_neighbors }\OperatorTok{=} \DecValTok{23}\NormalTok{,}
\NormalTok{                                 weights }\OperatorTok{=}\StringTok{\textquotesingle{}uniform\textquotesingle{}}\NormalTok{).fit(Xbin\_train,ybin\_train)}
\NormalTok{ybin\_pred }\OperatorTok{=}\NormalTok{ KNN\_modelbin.predict(Xbin\_test)}
\BuiltInTok{print}\NormalTok{(classification\_report(ybin\_test,ybin\_pred,target\_names}\OperatorTok{=}\NormalTok{[}\StringTok{"Mild"}\NormalTok{,}\StringTok{"Mod+Sev"}\NormalTok{]))}
\BuiltInTok{print}\NormalTok{(}\SpecialStringTok{f\textquotesingle{}Balanced Accuracy Score : }\SpecialCharTok{\{}\NormalTok{balanced\_accuracy\_score(ybin\_test,ybin\_pred)}\SpecialCharTok{\}}\SpecialStringTok{\textquotesingle{}}\NormalTok{)}
\end{Highlighting}
\end{Shaded}
\begin{Shaded}
\begin{Highlighting}[]
\NormalTok{ConfusionMatrixDisplay(confusion\_matrix(ybin\_test,ybin\_pred),}
\NormalTok{                       display\_labels}\OperatorTok{=}\NormalTok{[}\StringTok{"Mild"}\NormalTok{,}\StringTok{"Mod+Sev"}\NormalTok{]).plot()}\OperatorTok{;}
\NormalTok{plt.savefig(}\StringTok{"figure/knn\_bin\_conf.png"}\NormalTok{)}
\end{Highlighting}
\end{Shaded}
\begin{Shaded}
\begin{Highlighting}[]
\NormalTok{roc\_bin(KNN\_modelbin)}
\end{Highlighting}
\end{Shaded}
\normalsize

\hypertarget{rassal-ormanlar-modeli-kurulmasux131-1}{%
\section{Rassal Ormanlar Modeli Kurulması}\label{rassal-ormanlar-modeli-kurulmasux131-1}}

\scriptsize
\begin{Shaded}
\begin{Highlighting}[]
\NormalTok{param\_grid }\OperatorTok{=}\NormalTok{ \{}\StringTok{"n\_estimators"}\NormalTok{:np.arange(}\DecValTok{350}\NormalTok{,}\DecValTok{700}\NormalTok{,}\DecValTok{50}\NormalTok{),}
              \StringTok{"max\_features"}\NormalTok{:[}\StringTok{"auto"}\NormalTok{,}\StringTok{"sqrt"}\NormalTok{,}\StringTok{"log2"}\NormalTok{],}
              \StringTok{"ccp\_alpha"}\NormalTok{:[}\FloatTok{0.01}\NormalTok{,}\FloatTok{0.05}\NormalTok{,}\FloatTok{.1}\NormalTok{,}\FloatTok{0.3}\NormalTok{,}\FloatTok{.5}\NormalTok{],}
              \StringTok{"max\_samples"}\NormalTok{:np.arange(}\DecValTok{1}\NormalTok{,X\_train.shape[}\DecValTok{1}\NormalTok{],}\DecValTok{1}\NormalTok{)\}}
\NormalTok{GSC }\OperatorTok{=}\NormalTok{ GridSearchCV(RandomForestClassifier(random\_state}\OperatorTok{=}\DecValTok{13}\NormalTok{,criterion}\OperatorTok{=}\StringTok{"gini"}\NormalTok{),}
\NormalTok{                   param\_grid}\OperatorTok{=}\NormalTok{param\_grid,cv}\OperatorTok{=}\DecValTok{10}\NormalTok{,verbose}\OperatorTok{=}\DecValTok{1}\NormalTok{,n\_jobs}\OperatorTok{={-}}\DecValTok{1}\NormalTok{,scoring}\OperatorTok{=}\StringTok{"accuracy"}\NormalTok{)}
\NormalTok{GSC.fit(Xbin\_train,ybin\_train)                   }
\NormalTok{pd.DataFrame(GSC.cv\_results\_).to\_csv(}\StringTok{"data/RF\_bin\_Grid\_Res.csv"}\NormalTok{,index}\OperatorTok{=}\VariableTok{False}\NormalTok{)}
\NormalTok{rf\_bin }\OperatorTok{=}\NormalTok{ pd.read\_csv(}\StringTok{"data/RF\_bin\_Grid\_Res.csv"}\NormalTok{,sep}\OperatorTok{=}\StringTok{";"}\NormalTok{)}
\NormalTok{maxed }\OperatorTok{=}\NormalTok{ rf\_bin.groupby(}\StringTok{"rank\_test\_score"}\NormalTok{).}\BuiltInTok{max}\NormalTok{()}
\NormalTok{maxed }\OperatorTok{=}\NormalTok{ maxed[maxed[}\StringTok{"mean\_test\_score"}\NormalTok{] }\OperatorTok{\textgreater{}} \FloatTok{0.75}\NormalTok{]}
\NormalTok{plt.figure(figsize}\OperatorTok{=}\NormalTok{(}\DecValTok{15}\NormalTok{,}\DecValTok{10}\NormalTok{),dpi}\OperatorTok{=}\DecValTok{100}\NormalTok{)}\OperatorTok{;}
\NormalTok{sns.lineplot(x}\OperatorTok{=}\StringTok{"param\_n\_estimators"}\NormalTok{,y}\OperatorTok{=}\StringTok{"mean\_test\_score"}\NormalTok{,hue}\OperatorTok{=}\StringTok{"param\_ccp\_alpha"}\NormalTok{,data}\OperatorTok{=}\NormalTok{maxed,}
\NormalTok{             err\_style}\OperatorTok{=}\VariableTok{None}\NormalTok{,palette}\OperatorTok{=}\StringTok{"husl"}\NormalTok{)}\OperatorTok{;}
\NormalTok{plt.xlabel(}\StringTok{"Ormandaki Ağaç Sayısı"}\NormalTok{)}\OperatorTok{;}
\NormalTok{plt.ylabel(}\StringTok{"Eğitim Verisi Doğruluk Skoru (\%)"}\NormalTok{)}\OperatorTok{;}
\NormalTok{plt.legend(title}\OperatorTok{=}\StringTok{"Öğrenme Düzeyi"}\NormalTok{)}\OperatorTok{;}
\NormalTok{plt.savefig(}\StringTok{"figure/RF\_bin\_Grid\_Graph.png"}\NormalTok{)}\OperatorTok{;}
\end{Highlighting}
\end{Shaded}
\begin{Shaded}
\begin{Highlighting}[]
\BuiltInTok{print}\NormalTok{(}\SpecialStringTok{f\textquotesingle{}En İyi Parametreler : }\SpecialCharTok{\{}\NormalTok{GSC}\SpecialCharTok{.}\NormalTok{best\_params\_}\SpecialCharTok{\}}\SpecialStringTok{\textquotesingle{}}\NormalTok{)}
\end{Highlighting}
\end{Shaded}
\begin{Shaded}
\begin{Highlighting}[]
\NormalTok{RF\_modelbin }\OperatorTok{=}\NormalTok{ RandomForestClassifier(ccp\_alpha }\OperatorTok{=} \FloatTok{0.01}\NormalTok{, }
\NormalTok{                                     max\_features }\OperatorTok{=} \StringTok{"auto"}\NormalTok{,}
\NormalTok{                                     max\_samples }\OperatorTok{=} \DecValTok{10}\NormalTok{,}
\NormalTok{                                     n\_estimators }\OperatorTok{=} \DecValTok{400}\NormalTok{).fit(Xbin\_train,ybin\_train)}
\NormalTok{ybin\_pred }\OperatorTok{=}\NormalTok{ RF\_modelbin.predict(Xbin\_test)}
\BuiltInTok{print}\NormalTok{(classification\_report(ybin\_test,ybin\_pred,target\_names}\OperatorTok{=}\NormalTok{[}\StringTok{"Mild"}\NormalTok{,}\StringTok{"Mod+Sev"}\NormalTok{]))}
\BuiltInTok{print}\NormalTok{(}\SpecialStringTok{f\textquotesingle{}Balanced Accuracy Score : }\SpecialCharTok{\{}\NormalTok{balanced\_accuracy\_score(ybin\_test,ybin\_pred)}\SpecialCharTok{\}}\SpecialStringTok{\textquotesingle{}}\NormalTok{)}
\end{Highlighting}
\end{Shaded}
\begin{Shaded}
\begin{Highlighting}[]
\NormalTok{ConfusionMatrixDisplay(confusion\_matrix(ybin\_test,ybin\_pred),}
\NormalTok{                       display\_labels}\OperatorTok{=}\NormalTok{[}\StringTok{"Mild"}\NormalTok{,}\StringTok{"Mod+Sev"}\NormalTok{]).plot()}\OperatorTok{;}
\NormalTok{plt.savefig(}\StringTok{"figure/rf\_bin\_conf.png"}\NormalTok{)}\OperatorTok{;}
\end{Highlighting}
\end{Shaded}
\begin{Shaded}
\begin{Highlighting}[]
\NormalTok{roc\_bin(RF\_modelbin)}
\end{Highlighting}
\end{Shaded}
\normalsize

\hypertarget{xgboost-modeli-kurulmasux131-1}{%
\section{XGBoost Modeli Kurulması}\label{xgboost-modeli-kurulmasux131-1}}

\scriptsize
\begin{Shaded}
\begin{Highlighting}[]
\NormalTok{param\_grid }\OperatorTok{=}\NormalTok{ \{}\StringTok{"eta"}\NormalTok{:np.arange(}\DecValTok{0}\NormalTok{,}\FloatTok{.5}\NormalTok{,}\FloatTok{0.1}\NormalTok{),}
              \StringTok{"min\_child\_weight"}\NormalTok{:np.arange(}\DecValTok{1}\NormalTok{,}\DecValTok{11}\NormalTok{,}\DecValTok{1}\NormalTok{),}
              \StringTok{"max\_depth"}\NormalTok{:np.arange(}\DecValTok{3}\NormalTok{,}\DecValTok{11}\NormalTok{,}\DecValTok{1}\NormalTok{),}
              \StringTok{"n\_estimators"}\NormalTok{:np.arange(}\DecValTok{0}\NormalTok{,}\DecValTok{500}\NormalTok{,}\DecValTok{50}\NormalTok{)\}}
\NormalTok{GSC }\OperatorTok{=}\NormalTok{ GridSearchCV(XGBClassifier(eval\_metric}\OperatorTok{=}\StringTok{"auc"}\NormalTok{,use\_label\_encoder}\OperatorTok{=}\VariableTok{False}\NormalTok{,booster}\OperatorTok{=}\StringTok{"gbtree"}\NormalTok{),}
\NormalTok{                                param\_grid}\OperatorTok{=}\NormalTok{param\_grid,cv}\OperatorTok{=}\DecValTok{10}\NormalTok{,verbose}\OperatorTok{=}\DecValTok{1}\NormalTok{,n\_jobs}\OperatorTok{={-}}\DecValTok{1}\NormalTok{,scoring}\OperatorTok{=}\StringTok{"accuracy"}\NormalTok{)}
\NormalTok{GSC.fit(Xbin\_train,ybin\_train)                                }
\NormalTok{pd.DataFrame(GSC.cv\_results\_).to\_csv(}\StringTok{"data/XGB\_bin\_Grid\_Res.csv"}\NormalTok{,index}\OperatorTok{=}\VariableTok{False}\NormalTok{)}
\NormalTok{XGB\_bin }\OperatorTok{=}\NormalTok{ pd.read\_csv(}\StringTok{"data/XGB\_bin\_Grid\_Res.csv"}\NormalTok{,sep}\OperatorTok{=}\StringTok{";"}\NormalTok{)}
\NormalTok{maxed }\OperatorTok{=}\NormalTok{ XGB\_bin.groupby(}\StringTok{"rank\_test\_score"}\NormalTok{).}\BuiltInTok{max}\NormalTok{()}
\NormalTok{maxed }\OperatorTok{=}\NormalTok{ maxed[maxed[}\StringTok{"mean\_test\_score"}\NormalTok{] }\OperatorTok{\textgreater{}} \FloatTok{0.75}\NormalTok{]}
\NormalTok{plt.figure(figsize}\OperatorTok{=}\NormalTok{(}\DecValTok{15}\NormalTok{,}\DecValTok{10}\NormalTok{),dpi}\OperatorTok{=}\DecValTok{100}\NormalTok{)}\OperatorTok{;}
\NormalTok{sns.lineplot(x}\OperatorTok{=}\StringTok{"param\_n\_estimators"}\NormalTok{,y}\OperatorTok{=}\StringTok{"mean\_test\_score"}\NormalTok{,hue}\OperatorTok{=}\StringTok{"param\_eta"}\NormalTok{,data}\OperatorTok{=}\NormalTok{maxed,}
\NormalTok{             err\_style}\OperatorTok{=}\VariableTok{None}\NormalTok{,palette}\OperatorTok{=}\StringTok{"husl"}\NormalTok{)}\OperatorTok{;}
\NormalTok{plt.xlabel(}\StringTok{"Tahminleyici Sayısı"}\NormalTok{)}\OperatorTok{;}
\NormalTok{plt.ylabel(}\StringTok{"Eğitim Verisi Doğruluk Skoru (\%)"}\NormalTok{)}\OperatorTok{;}
\NormalTok{plt.legend(title}\OperatorTok{=}\StringTok{"Öğrenme Düzeyi"}\NormalTok{)}\OperatorTok{;}
\NormalTok{plt.savefig(}\StringTok{"figure/XGB\_bin\_Grid\_Graph.png"}\NormalTok{)}\OperatorTok{;}
\end{Highlighting}
\end{Shaded}
\begin{Shaded}
\begin{Highlighting}[]
\BuiltInTok{print}\NormalTok{(}\SpecialStringTok{f\textquotesingle{}En İyi Parametreler : }\SpecialCharTok{\{}\NormalTok{GSC}\SpecialCharTok{.}\NormalTok{best\_params\_}\SpecialCharTok{\}}\SpecialStringTok{\textquotesingle{}}\NormalTok{)}
\end{Highlighting}
\end{Shaded}
\begin{Shaded}
\begin{Highlighting}[]
\NormalTok{XGB\_modelbin }\OperatorTok{=}\NormalTok{ XGBClassifier(eval\_metric}\OperatorTok{=}\StringTok{"auc"}\NormalTok{,}
\NormalTok{                             use\_label\_encoder}\OperatorTok{=}\VariableTok{False}\NormalTok{,}
\NormalTok{                             booster}\OperatorTok{=}\StringTok{"gbtree"}\NormalTok{,}
\NormalTok{                             eta}\OperatorTok{=}\FloatTok{0.2}\NormalTok{,}
\NormalTok{                             max\_depth}\OperatorTok{=}\DecValTok{10}\NormalTok{,}
\NormalTok{                             min\_child\_weight}\OperatorTok{=}\DecValTok{10}\NormalTok{,}
\NormalTok{                             n\_estimators}\OperatorTok{=}\DecValTok{400}\NormalTok{).fit(Xbin\_train,ybin\_train)}
\NormalTok{ybin\_pred }\OperatorTok{=}\NormalTok{ XGB\_modelbin.predict(Xbin\_test)}
\BuiltInTok{print}\NormalTok{(classification\_report(ybin\_test,ybin\_pred,target\_names}\OperatorTok{=}\NormalTok{[}\StringTok{"Mild"}\NormalTok{,}\StringTok{"Mod+Sev"}\NormalTok{]))}
\BuiltInTok{print}\NormalTok{(}\SpecialStringTok{f\textquotesingle{}Balanced Accuracy Score : }\SpecialCharTok{\{}\NormalTok{balanced\_accuracy\_score(ybin\_test,ybin\_pred)}\SpecialCharTok{\}}\SpecialStringTok{\textquotesingle{}}\NormalTok{)}
\end{Highlighting}
\end{Shaded}
\begin{Shaded}
\begin{Highlighting}[]
\NormalTok{ConfusionMatrixDisplay(confusion\_matrix(ybin\_test,ybin\_pred),}
\NormalTok{                       display\_labels}\OperatorTok{=}\NormalTok{[}\StringTok{"Mild"}\NormalTok{,}\StringTok{"Mod+Sev"}\NormalTok{]).plot()}\OperatorTok{;}
\NormalTok{plt.savefig(}\StringTok{"figure/XGB\_bin\_conf.png"}\NormalTok{)}
\NormalTok{plt.close(}\StringTok{\textquotesingle{}all\textquotesingle{}}\NormalTok{)}
\end{Highlighting}
\end{Shaded}
\begin{Shaded}
\begin{Highlighting}[]
\NormalTok{roc\_bin(XGB\_modelbin)}
\NormalTok{plt.close(}\StringTok{\textquotesingle{}all\textquotesingle{}}\NormalTok{)}
\end{Highlighting}
\end{Shaded}
\normalsize

\hypertarget{yapay-sinir-aux11flarux131-modeli-kurulmasux131-1}{%
\section{Yapay Sinir Ağları Modeli Kurulması}\label{yapay-sinir-aux11flarux131-modeli-kurulmasux131-1}}

\scriptsize
\begin{Shaded}
\begin{Highlighting}[]
\NormalTok{params }\OperatorTok{=}\NormalTok{ \{}\StringTok{"hidden\_layer\_sizes"}\NormalTok{:np.arange(}\DecValTok{1}\NormalTok{,}\DecValTok{26}\NormalTok{,}\DecValTok{1}\NormalTok{),}
          \StringTok{"learning\_rate"}\NormalTok{:[}\StringTok{"adaptive"}\NormalTok{,}\StringTok{"constant"}\NormalTok{,}\StringTok{"invscaling"}\NormalTok{],}
          \StringTok{"activation"}\NormalTok{:[}\StringTok{"identity"}\NormalTok{,}\StringTok{"logistic"}\NormalTok{,}\StringTok{"tanh"}\NormalTok{,}\StringTok{"relu"}\NormalTok{]\}}
\NormalTok{grid2 }\OperatorTok{=}\NormalTok{ GridSearchCV(MLPClassifier(random\_state}\OperatorTok{=}\DecValTok{13}\NormalTok{),param\_grid}\OperatorTok{=}\NormalTok{params,cv}\OperatorTok{=}\DecValTok{10}\NormalTok{,verbose}\OperatorTok{=}\DecValTok{1}\NormalTok{,}
\NormalTok{                     n\_jobs}\OperatorTok{={-}}\DecValTok{1}\NormalTok{,scoring}\OperatorTok{=}\StringTok{"accuracy"}\NormalTok{).fit(Xbin\_train,ybin\_train)}
\NormalTok{pd.DataFrame(grid2.cv\_results\_).to\_csv(}\StringTok{"data/NN\_bin\_Grid\_Res.csv"}\NormalTok{,index}\OperatorTok{=}\VariableTok{False}\NormalTok{)}
\NormalTok{nn\_bin }\OperatorTok{=}\NormalTok{ pd.read\_csv(}\StringTok{"data/NN\_bin\_Grid\_Res.csv"}\NormalTok{,sep}\OperatorTok{=}\StringTok{";"}\NormalTok{)}
\NormalTok{plt.figure(figsize}\OperatorTok{=}\NormalTok{(}\DecValTok{15}\NormalTok{,}\DecValTok{10}\NormalTok{),dpi}\OperatorTok{=}\DecValTok{100}\NormalTok{)}\OperatorTok{;}
\NormalTok{sns.lineplot(x}\OperatorTok{=}\NormalTok{nn\_bin[}\StringTok{"param\_hidden\_layer\_sizes"}\NormalTok{],y}\OperatorTok{=}\NormalTok{nn\_bin[}\StringTok{"mean\_test\_score"}\NormalTok{]}\OperatorTok{*}\DecValTok{100}\NormalTok{,}
\NormalTok{             hue}\OperatorTok{=}\NormalTok{nn\_bin[}\StringTok{"param\_activation"}\NormalTok{])}\OperatorTok{;}
\NormalTok{plt.xlabel(}\StringTok{"Hidden Layer Sizes"}\NormalTok{)}\OperatorTok{;}
\NormalTok{plt.ylabel(}\StringTok{"Eğitim Verisi Doğruluk Skoru (\%)"}\NormalTok{)}\OperatorTok{;}
\NormalTok{plt.legend(title}\OperatorTok{=}\StringTok{"Aktivasyon Fonksiyonu"}\NormalTok{)}\OperatorTok{;}
\NormalTok{plt.savefig(}\StringTok{"figure/NN\_bin\_Grid\_Graph.png"}\NormalTok{)}\OperatorTok{;}
\end{Highlighting}
\end{Shaded}
\begin{Shaded}
\begin{Highlighting}[]
\BuiltInTok{print}\NormalTok{(}\SpecialStringTok{f\textquotesingle{}En İyi Parametreler : }\SpecialCharTok{\{}\NormalTok{grid2}\SpecialCharTok{.}\NormalTok{best\_params\_}\SpecialCharTok{\}}\SpecialStringTok{\textquotesingle{}}\NormalTok{)}
\end{Highlighting}
\end{Shaded}
\begin{Shaded}
\begin{Highlighting}[]
\NormalTok{NN\_modelbin }\OperatorTok{=}\NormalTok{ MLPClassifier(hidden\_layer\_sizes}\OperatorTok{=}\DecValTok{19}\NormalTok{,}
\NormalTok{                            learning\_rate}\OperatorTok{=}\StringTok{\textquotesingle{}adaptive\textquotesingle{}}\NormalTok{,}
\NormalTok{                            random\_state}\OperatorTok{=}\DecValTok{13}\NormalTok{,}
\NormalTok{                            max\_iter}\OperatorTok{=}\DecValTok{3000}\NormalTok{).fit(Xbin\_train,ybin\_train)}
\NormalTok{ybin\_pred }\OperatorTok{=}\NormalTok{ NN\_modelbin.predict(Xbin\_test)}
\BuiltInTok{print}\NormalTok{(classification\_report(ybin\_test,ybin\_pred,target\_names}\OperatorTok{=}\NormalTok{[}\StringTok{"Mild"}\NormalTok{,}\StringTok{"Mod+Sev"}\NormalTok{]))}
\BuiltInTok{print}\NormalTok{(}\SpecialStringTok{f\textquotesingle{}Balanced Accuracy Score : }\SpecialCharTok{\{}\NormalTok{balanced\_accuracy\_score(ybin\_test,ybin\_pred)}\SpecialCharTok{\}}\SpecialStringTok{\textquotesingle{}}\NormalTok{)}
\end{Highlighting}
\end{Shaded}
\begin{Shaded}
\begin{Highlighting}[]
\NormalTok{ConfusionMatrixDisplay(confusion\_matrix(ybin\_test,ybin\_pred),}
\NormalTok{                       display\_labels}\OperatorTok{=}\NormalTok{[}\StringTok{"Mild"}\NormalTok{,}\StringTok{"Mod+Sev"}\NormalTok{]).plot()}\OperatorTok{;}
\NormalTok{plt.savefig(}\StringTok{"figure/nn\_bin\_conf.png"}\NormalTok{)}
\end{Highlighting}
\end{Shaded}
\begin{Shaded}
\begin{Highlighting}[]
\NormalTok{roc\_bin(NN\_modelbin)}
\end{Highlighting}
\end{Shaded}
\normalsize



\end{document}
